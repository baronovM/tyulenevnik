\subsection{Формулы Тейлора с остаточными членами в форме Лагранжа и Пеано}
\begin{note}
    Считаем $dx = (dx_1, ..., dx_n)$ -- фиксированным вектором;
    \[A = dx_1 \dfrac{\partial}{\partial x_{1}} + ... + dx_n \dfrac{\partial}{\partial x_{n}}\] -- линейный оператор дифференцирования;
    \[A[f](x) = d_x f(dx)\]
    % Тут я ещё на лекцию не пришёл, пока без пояснений
\end{note}

\begin{lemma}
    Если $f$: $\Omega \rightarrow \R$, где $\Omega \subset \R^n, \neq \emptyset, Int \Omega = \Omega$ -- $k-$раз дифференцируема в $\forall$ т. $x_0$, то $\forall dx \in \R^n A^k[f](x^0)(dx) = d^k_{x_0}[f](dx)$
\end{lemma}
\begin{proof}
    Доказательство очевидно, предполагается читателю в качестве упражнения (Спойлер: по индукции).
\end{proof}
\begin{theorem}[Формула Тейлора с остаточным членом в форме Лагранжа]
    Пусть $m \in \N_0$ и $f: \Omega \rightarrow \R$, где $\Omega \subset \R^n, \neq \emptyset, Int \Omega = \Omega$(Открытое и непустое) и $f \in \DIF^{m + 1}(B_{\delta}(x^{0}))$ при каком-то $\delta > 0$. Тогда, для любого $x \in B_{\delta}(x^{0})$ справедлива следующая формула: \[\exists \theta \in (0, 1): f(x) = f(x^0) + \sum\limits_{k = 1}^{m}\frac{d^k_{x^0}(dx)}{k!} + \frac{d^{m + 1}_{x^{\theta}}(dx)}{(m + 1)!}, \text{ где }dx = x - x^0, x^{\theta} = x^0 + \theta(x - x^0)\]
\end{theorem}
\begin{proof}
    Зафиксируем точку $x \in B_{\delta}(x^0)$. Пусть $x^t = x^0 + t(x - x^0)$. Тогда функцией $\phi$ будем называть $\phi(t) = f(x^t) = f(x^0 + t(x - x^0)$. Заметим, что \[\dfrac{d}{dt} = \sum\limits_{i = 1}^n \dfrac{\partial f}{\partial x_i}(x^t) \cdot (x_i - x^0_i) = d_{x^t}f(dx)\]
    Тогда, $\forall s \in \overline{1, m + 1}$ будет выполнено следующее равенство: \[\dfrac{d^s \phi}{d t^s}(t) = d^s_{x^t} f(dx)\]
    Докажем это утверждение по индукции. Истинность базы была показана выше, докажем переход от $s$ к $s + 1$:
    \[\dfrac{d^{s + 1}}{dt^{s + 1}} \phi(t) = \dfrac{d}{dt}(d^s_{x^t} f(dx)) = \dfrac{d}{dt} (\sum\limits_{i_1 = 1}^n\sum ... \sum\limits_{i_s = 1}^n \dfrac{\partial^sf}{\partial x_{i_1} \cdot ... \cdot \partial x_{i_s}}(x^t)dx_{i_1}\cdot ... \cdot dx_{i_s}) =\]
    \[= \sum\limits_{i_1 = 1}^n \cdot ... \cdot \sum\limits_{i_{s + 1} = 1}^n \dfrac{\partial^{s + 1}f}{\partial x_{i_1}\cdot ... \cdot \partial x_{i_{s + 1}}}dx_{i_1}\cdot...\cdot dx_{i_{s+1}} = d^{s + 1}_{x^t} f(dx)\]
    Значит, к $\phi$ можно применить одномерную формулу Тейлора с остаточным членом в форме Лагранжа:
    \[\phi(1) = \phi(0) + \sum\limits_{k = 1}^m \frac{1}{k!} \dfrac{d^k \phi}{dt^k}(0) + \frac{1}{(m + 1)!} \dfrac{d^{m + 1}\phi}{dt^{m + 1}}(\theta) \cdot 1\]
    \[f(x) = f(x^0) + \sum\limits_{i = 1}^m \frac{1}{i!} d_{x^0}^if(dx) + \frac{1}{(m + 1)!}d_{x^\theta}^{m + 1}f(dx)\]
\end{proof}
\begin{theorem}[Формула Тейлора с остаточным членом в форме Пеано]
    Пусть $f$: $\Omega \rightarrow \R, m \in \N$ и $\exists \delta > 0, $ такой что $f \in C^m(B_\delta(x^0))$ (f -- m-гладкая). Тогда, $\forall x \in B_{\delta}(x^0):$ 
    \[f(x) = f(x^0) + \sum\limits_{i = 1}^n \frac{1}{i!}d^k_{x^0} f(dx) + o(||x - x^0||^m), x \rightarrow x^0\]
\end{theorem}
\begin{proof}
    Все частные производные порядка $m$~---~непрерывны в $B_{\delta}(x^0)$, значит, по достаточному условию дифференцируемости, все частные производные $m - 1$ порядка являются дифференцируемыми функциями в этом шаре. $\Rightarrow f \in \DIF^m(B_{\delta}(x^m) \Rightarrow$ по формуле Тейлора с остаточным членом в форме Лагранжа:
    \[\forall x \in B_{\delta}(x^0) \exists \theta \in (0, 1):\]
    \[f(x) = f(x^0) + \sum\limits_{k = 1}^{m - 1} \dfrac{d_{x^0}^k f(dx)}{k!} + \dfrac{1}{m!} d_{x^\theta}^m f(dx) = (*), x^\theta = x^0 + \theta(x - x^0)\]
    \[(*) = f(x^0) + \sum\limits_{k = 1}^m \dfrac{d_{x^0}^k f(dx)}{k!} + R, R = \dfrac{1}{m!}[d_{x^\theta}^m f(dx) - d_{x^0}^m f(dx)]\]
    Покажем, что $R = o(||x - x^0||), x \rightarrow x^0$. Тогда по неравенству треугольника
    \[|R| \leq \dfrac{1}{m!}\sum\limits_{i_1 = 1}^n \cdot ... \cdot \sum\limits_{i_m = 1}^n| \dfrac{\partial^m f}{\partial x_{i_1} \cdot ... \cdot \partial x_{i_n}}(x^\theta) - \dfrac{\partial^m f}{\partial x_{i_1} \cdot ... \cdot \partial x_{i_n}}(x^0)| \cdot\]
    \[\cdot |dx_{i_1} \cdot ... \cdot dx_{i_n}| \leq (*)\]
    В тоже время, из непрерывности частных производных $m-$го порядка следует
    \[\varepsilon_{x^0}(x) = \max\limits_{i_1, ..., i_n} |\dfrac{\partial^m f(x^0)}{\partial x_{i_m} \cdot ... \cdot \partial x_{i_1}} - \dfrac{\partial^m f(x^0)}{\partial x_{i_m} \cdot ... \cdot \partial x_{i_1}}| \rightarrow 0, x \rightarrow x^0\]
    Поскольку $x^\theta \xrightarrow{x \rightarrow x^0} x^0 \Rightarrow |dx_{i_1}\cdot ... \cdot dx_{i_m}| \leq ||x - x^\theta||^m$, поскольку $|dx_{i_j}| = |x_j^0 - x_j| = \sqrt{(x^0_j - x_j)^2} \leq ||x - x^0||$. 
    \[(*) \leq \dfrac{m^n}{m!} \varepsilon_{x^0} (x) ||x - x^0||^m = o(||x - x^0||^m), x \rightarrow x^0\].
\end{proof}
\begin{note}
    Справедлива теорема о единственности: если $f \in C^m(B_\delta(x^0)) \Rightarrow f(x) = P_m(dx) + o(||x - x^0||^m), x \rightarrow x^0 \Rightarrow P_m(dx)$ совпадает $\sum\limits_{k = 0}^m \dfrac{1}{k!}  d_{x^0}^k f(dx)$, то есть с полиномом Тейлора.
\end{note}
\subsection{Теоремы об открытом отображении и локальном диффеоморфизме}
\begin{lemma}
    Пусть $A \in \mathcal{L}(\R^m, \R^m)$,  $A$~---~обратимое. Тогда $\exists c > 0$: $\|A(x) - A(y)\| \geq c\|x - y\|$
\end{lemma}
\begin{proof}
    Пусть $A$~---~обратимо. $\exists A^{-1} \Rightarrow x = A^{-1}(A(x)) \Rightarrow ||x|| = ||A^{-1}(A(x))|| \leq ||A^{-1}||||A(x)|| \Rightarrow ||A(x)|| \geq \dfrac{1}{||A^{-1}||} ||x|| \Rightarrow$ в силу линейности отображения верно, что $\forall x, y \in \R^m$: $\|A(x - y)\| = \|A(x) - A(y)\| \geq \dfrac{1}{||A^{-1}||}||x - y||$.
\end{proof}
\begin{theorem}
    Пусть $F$: $\Omega \rightarrow \R^m, \Omega \subset \R^m$~---~дифференцируема в точке $x^0$, и матрица Якоби отображения $DF(x^0)$~---~обратима. Тогда $\exists \delta > 0$ и $\exists c > 0$: $\forall x \in \overline{B}_\delta(x^0) \Rightarrow \|F(x) - F(x^0)\| \geq c \|x - x^{0}\|$.
\end{theorem}
\begin{proof}
    Поскольку $F$~---~дифференцируемо в точке $x^0 \Rightarrow$
    \[F(x) = F(x^0) + DF(x^0)(x - x^0) + \varepsilon_{x^0}\|x - x^0\|\]
    По предыдущей лемме, $DF(x^0)(x - x^0) \geq 2C \cdot \|x - x^0\|, 2C = \dfrac{1}{\|DF(x^0)^{-1}\|}$
    Тогда по неравенству треугольника: \[\|F(x) - F(x^0)\| \geq \|DF(x^0)(x - x^0)\| - \big{\|} \varepsilon_{x^0}(x)\|x - x^0\|\big{\|} \geq (*)\]
    Но $\big{\|} \varepsilon_{x^0}(x)\| x - x^0 \| \big{\|} \rightarrow 0, x \rightarrow x^0$, значит $\exists \delta > 0$: $\|\varepsilon_{x^0}(x) \leq \dfrac{1}{2\|(DF(x^0))^{-1}\|}\|$. Поэтому
    \[(*) \geq \dfrac{1}{\|(DF(x^0))^{-1}\|} \|x - x^0 \| - \dfrac{\|x - x^0\|}{2\|(DF(x^0))^{-1}\|} = c\|x - x^0\|\]
    И это верно для $\forall x \in B_\delta(x^0)$. Из нестрогости неравенства $\Rightarrow$ справедливость неравенства и для $\overline{B}_\delta(x^0)$. (Небольшое замечание: всё это время мы работаем с $\overline{B}_\delta(x^0) \cap \Omega$)
\end{proof}
\begin{theorem}[Теорема об открытом отображении]
    Пусть $F: \Omega \rightarrow \R^m, F \in \DIF(\Omega, \R^m)$ и пусть $\forall x^0 \in \Omega \hookrightarrow DF(x^0)$~---~обратима, $\Rightarrow$ образ открытого открыт и $F(\Omega)$~---~открыто в $\R^m$
\end{theorem}
\begin{proof}
    Фиксируем $x^0 \in \Omega$ и $y^0 = F(x^0)$. Покажем, что $\exists r > 0$ такое что $B_r(y^0) \subset F(\Omega)$: 
    Пользуясь предыдущей теоремой $\exists \delta > 0$ и $c > 0$: $\overline{B}_\delta \subset \Omega$ и $\|F(x) - F(x^0)\| \geq c \|x - x^0\| \forall x \in \overline{B}_\delta(x^0)$. Покажем что $r = \dfrac{c\delta}{2}$ является искомым. Возьмём произвольную $y^* \in B_\delta(y^0)$ и покажем что $\exists x^* \in B_\delta(x^0)$: $F(x^*) = y^*$. Заметим, что
    \[\forall x \text{ на сфере }\|F(x) - F(x^0)\| \geq c\delta \]
    \[\|F(x) - y^*\| \geq \|F(x) - y^*\| - \|y^0 - y^*\| \geq c\delta - \|y^0 - y^*\| > \dfrac{c\delta}{2}\]
    Рассмотрим функцию $h(x) = \|F(x) - y^*\|$. Тогда, $h(x^0) < \dfrac{c\delta}{2}$, а с другой стороны $\forall x \in B_\delta(x^0) h(x) > \dfrac{c\delta}{2}$. В то же время, поскольку $h \in C$ как функция от $x \Rightarrow$ из компактности $\overline{B}_\delta(x^0)$ следует что $h(x)$ достигает своё наименьшее значение и минимум достигается во внутренней точке. Из неотрицательности нормы следует истинность данного утверждения и для $h^2(x)$. Значит $\exists x^* \in B_\delta(x^0)$, в которой $h^2(x)$ принимает минимальное значение. Это внутренняя точка шара $\Rightarrow$ в ней $grad h \neq 0$
    \[\dfrac{\partial h^2(x)}{\partial x_i} \dfrac{\partial}{\partial x_i}(\|F(x) - y^*\|^2) = \]
    \[= \dfrac{\partial}{\partial x_i}(\sum\limits_{j = 1}^m(F_j(x) - y_i^*)^2 = \sum\limits_{j = 1}^m 2(F_j(x) - y^*_j) \dfrac{\partial F}{\partial x_i}(x) \Rightarrow \text{ в точке }x^*:\]
    \[\sum\limits_{j = 1}^m 2(F_j(x) - y_i^*)\dfrac{\partial F}{\partial x_i}(x^*) \Rightarrow\]
    \[grad h^2(x^*) = 2DF^T(x^*)(F(x^*) - y^*) \Rightarrow\]
    \[F(x^*) - y^* = 0\]
    Что и требовалось доказать.
\end{proof}
%Используй лучше ~---~ для тире (пробелы до и после ~ делать не надо)
%для ... есть \ldots
%также из мелочей можешь пж разделять $$: $$, то есть тип чтобы двоеточие с пробелом было вне формулы (если в двойных $$, то есть тип по центру формула, то \text{: }), так отступ другой оно делает