\newpage
\section{Анекдоты}
\begin{enumerate}
    \item Ну и тут возникает вопрос, как это(систему множеств) пощупать? \\
    Как говорил мой коллега в Италии: No Chance \\
    Потому что мы с вами не боги, а всего лишь люди
    \item Как я возьму множество всех сигма алгебр содержащих $\E$ Ну это из разряда давайте прыгнем на Луну. Ну или из разряда, как в моем любимом фильме: \\
    - Видишь суслика? \\
    - Нет \\
    - И я нет \\
    - А он есть \\
    \item Вот и всё. Понятно что произошло? \\
    Чет как-то вы подсдулись немножко... \\
    Ну нет, я понимаю что это сложно, для первого восприятия это правда сложно. Я сам когда читал в первый раз, я обалдевал, как очень все хитро получается. Кажется что здесь где-то подвох зашит. Но на самом деле нет
    \item - Поняли что сказал? Нет? \\
    Ну кто не понял пересмотрите на видеозаписи и все станет понятно

    \item Ну если кто сейчас не понял с первого раза. Потом замедленно это пересмотрите и убедитесь, что я говорю очень простые вещи

    \item (Теорема Тонелли) смотрим, получаем удовольствие, да?)

    \item Что-то народ стал убегать, отключаться... что вы отключаетесь то?

    \item Все я уже зарапортовался... все разобрались

    \item Такое множество называется множеством Витали, которое названо в честь Болонского математика Джузеппе Витали. При чем Витали - это не погоняло, это фамилия\\ Сама Болония $-$ прекрасный итальянский городок, генерал Тюленев любит туда летать. Эту часть сказали убрать из теха. Видимо не любил туда летать :(((

    \item Хорошо, ну давайте переходим к поверхностным интегралам. Сейчас жизнь станет полегче, потому что я не буду давать общий интеграл по многообразию. ... Сейчас в конце будет спуск с горы)

    \item (параметризуем сферу) .. этот корень будет иметь бесконечную производную, чтобы сделать параметризацию кусочно гладкой поднимем одну шляпку на уровень Италии, а вторую опустим н уровень Аргентины
\end{enumerate}