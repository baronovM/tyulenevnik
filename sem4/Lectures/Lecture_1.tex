\subsection{Полнота пространств $L_p$}

\noindent

$(X, \MM, \mu)$ ~---~ п-во с мерой.

$p \in [1, +\infty]$

$\widetilde L_p (\mu)$  ~---~ полунормированное линейное пространство. Лишь \textit{полу}нормированное потому, что равенство 0 интеграла в p-ой степени от функции не означает равенство 0 этой функции, а лишь равенство этой функции нулю почти всюду.

$L_p (\mu)$ ~---~ нормированное линейное пространство.

Это всё было в прошлом семестре, теперь же мы докажем полноту пространства $L_p$.

\begin{definition}
	Пусть $E = (E, \|\cdot\|)$  ~---~ л.н.п. Оно называется \textit{полным}, если
	
	$\forall$ фундаментальной (по норме $\|\cdot\|$) п-ть $\{x^n\}$ п-ва $E$ сходится по норме пространства $E$ к некоторому элементу $x \in E$.
\end{definition}

\begin{definition}
	Дано $E = (E, \|\cdot\|)$  ~---~ л.н.п. Пара последовательностей $\{x^n\}_{n=1}^{\infty}$ и 
\end{definition}

\begin{theorem}
	Критерий полноты.
\end{theorem}

