\begin{proof}
    Нужно проверить, что $D^{\alpha}\lambda \in D'(\R^n)$. \\
    Линейность по $\phi$ очевидна: оператор производной -- линеен. \\
    Проверим непрерывность.
    Пусть $\{\phi_m\} \subset D(\R^n)$ такая, что $\phi_m \xrightarrow{D} \phi, m \ra +\infty$.
    \[
        |(D^{\alpha} \lambda, \phi_m) - (D^{\alpha} \lambda, \phi)| = |(D^{\alpha} \lambda, \phi_m - \phi)| = |(\lambda, D^{\alpha}(\phi_m - \phi))|.
    \]
    В силу определения сходимости в $D$, очевидно, что $D^{\alpha} \phi_m \xrightarrow{D} D^{\alpha} \phi, m \ra +\infty$.
    Тогда, поскольку $\lambda \in D'(\R^n)$, то и $(\lambda, D^{\alpha}(\phi_m - \phi)) \ra 0, m \ra +\infty$. \\
    Пусть $\lambda$ -- регулярный функционал, который порождён $C^1$-гладкой $f_\lambda$.
    Покажем, что обобщенная производная совпадает с классической. То есть обобщенной производной будет соответствовать регулярный функционал, который поточечно почти всюду совпадает с классической частной производной. Пусть $D_j$ -- частная производная по $j$-й координате в обобщённом смысле. Тогда
    \[
        (D_j \lambda, \phi) = -(\lambda, D_j \phi) = -\int_{\R^n} f_{\lambda} \dfrac{\partial \phi}{\partial x_j}dx = (*)
    \]
    Далее будем работать с сечениями и, воспользовавшись теоремой Фубини, так как $f_{\lambda}$ - $C^1$-гладка, $\dfrac{\partial \phi}{\partial x_j}$ - бесконечно дифференцируема и финитна, получим
    \[
        (*) = -\int_{\R^{n - 1}} d\hat{x}_j \int_{\R} f_{\lambda}(x_1, \ldots, x_j, \ldots x_n) \dfrac{\partial \phi}{\partial x_j} dx_j.
    \]
    Где $\hat{x}_j$ - та у которой пропущена j-ая координата. При каждом фиксированном $\hat{x}_j$ проинтегрируем по частям по j-ой координате:
    \[
        \int_{\R} f_{\lambda}(x_1, \ldots) \dfrac{\partial \phi}{\partial x_j} dx_j = f_{\lambda}(\ldots)\phi(\ldots)|^{x_j=c}_{x_j=-c} - \int_{\R} \dfrac{\partial f_{\lambda}}{\partial x_j} \phi(x_1, \ldots) dx_j.
    \]
    При каждой $\hat{x}_j$:  $f_{\lambda}(\ldots)\phi(\ldots)|^{x_j=c}_{x_j=-c} = 0$. То есть 
    \[
        (*) = -\int_{\R^{n - 1}} (f_{\lambda}(\ldots)\phi(\ldots)|^{x_j=c}_{x_j=-c}) d\hat{x}_j + \int_{\R^{n - 1}} d\hat{x}_j \int_{\R} \dfrac{\partial f_{\lambda}}{\partial x_j} \phi(x_1, \ldots) dx_j .
    \]
    А это, с учётом внешнего интеграла и повторного применения теоремы Фубини, в точности равно $(\dfrac{\partial f_{\lambda}}{\partial x_j}, \phi)$ -- что и требовалось показать. \\
    Аналогично по индукции мы можем показать для производной произвольного порядка.
\end{proof}
\begin{note}
    Всяких $\lambda \in D'(\R^n)$ можно дифференцировать сколько угодно раз (в обобщённом смысле).
\end{note}
\begin{lemma}
    Если $\lambda_m \xrightarrow{D'} \lambda, m \ra +\infty$, то $\forall \alpha \in (\N_0)^n$ выполняется
    \[
        D^{\alpha} \lambda_m \xrightarrow{D'} D^{\alpha} \lambda, m \ra +\infty.`
    \]
\end{lemma}
\begin{proof}
    По определению:
    \[
        (D^{\alpha} \lambda_m, \phi) = (-1)^{|\alpha|} (\lambda_m, D^{\alpha} \phi) \ra (-1)^{|\alpha|}(\lambda, D^{\alpha}\phi) = (D^{\alpha} \lambda, \phi).
    \]
\end{proof}
\begin{theorem}[Правило Лейбница]
    Пусть $\lambda \in D'(\R)$, а $g \in C^\infty(\R)$.
    Тогда
    \[
        (g\lambda)' = g' \lambda + g \lambda',
    \]
    где равенства и производные понимаются в смысле $D'(\R)$.
\end{theorem}
\begin{proof}
    По определению и ранее доказанным свойствам $\forall \phi \in D(\R)$:
    \begin{multline*}
        ((g\lambda)', \phi) = -(g \lambda, \phi') = -(\lambda, g \phi') = -(\lambda, (g\phi)' - g'\phi) = \\ = -(\lambda, (g\phi)') + (\lambda, g' \phi) = (\lambda', g\phi) + (g'\lambda, \phi) = (g\lambda' + g'\lambda, \phi).
    \end{multline*}
    В силу произвольности $\phi$ мы и получаем утверждение теоремы.
\end{proof}
\subsection{Пространства $S$ и $S'$.}
\begin{definition}
    Определим пространство $S(\R^n)$ -- пространство всех бесконечно дифференцируемых функций, быстро убывающих на бесконечности, то есть:
    \[
        \forall l \in \N_0: \|\phi\|_l \equiv \sup\limits_{x \in \R^n} ((1 + |x|)^l \max\limits_{0 \leq |\alpha| \leq l} |D^{\alpha} \phi|) < +\infty.
    \]
    Вместе со следующей сходимостью:
        $\phi_l \xrightarrow{S} \phi, l \ra +\infty.$
    если $\forall l \in \N_0 \hookrightarrow \|\phi - \phi_m\|_l \ra 0, m \ra +\infty$.
\end{definition}
\begin{note}
    $\forall l \in \N_0$ верно, что $\|\cdot \|_l$ -- действительно норма.
\end{note}

\begin{note}[Примечание техающего]
Было в качестве упражнения, попробовал доказать сам.
\end{note}

\begin{proof}
Проверим три аксиомы нормы.

\medskip
\noindent 1. Неотрицательность и однозначность нуля.
По определению, \(\|\varphi\|_l\)~--- супремум неотрицательной функции
\[
x\;\mapsto\;(1+|x|)^l \max_{0 \le |\alpha|\le l}\bigl|D^\alpha\varphi(x)\bigr|\ge0,
\]
поэтому \(\|\varphi\|_l\ge0\). Если \(\|\varphi\|_l=0\), то
\[
\sup_{x\in\mathbb{R}^n}(1+|x|)^l\max_{0 \le |\alpha|\le l}|D^\alpha\varphi(x)|=0
\;\Longrightarrow\;
\forall x,\;\max_{0 \le |\alpha|\le l}|D^\alpha\varphi(x)|=0,
\]
т.~е. \(D^\alpha\varphi(x)\equiv0\) для всех \(0 \le |\alpha|\le l\). При \(\alpha=0\) это даёт \(\varphi\equiv0\).

\medskip
\noindent 2. Однородность.
Для любого \(a\in\mathbb{R}\) имеем
\[
\|a\varphi\|_l
=\sup_x(1+|x|)^l\max_{0 \le |\alpha|\le l}\bigl|D^\alpha(a\varphi)(x)\bigr|
=\sup_x(1+|x|)^l\max_{0 \le |\alpha|\le l}\bigl|a\,D^\alpha\varphi(x)\bigr|
=|a|\,\|\varphi\|_l.
\]

\medskip
\noindent 3. Неравенство треугольника.
Пусть \(\varphi,\psi\in S(\mathbb{R}^n)\). Тогда для каждого \(x\) и каждого \(|\alpha|\le l\)
\[
|D^\alpha(\varphi+\psi)(x)|
\le|D^\alpha\varphi(x)|+|D^\alpha\psi(x)|
\le\max_{0 \le |\alpha|\le l}|D^\alpha\varphi(x)|+\max_{0 \le |\alpha|\le l}|D^\alpha\psi(x)|.
\]
Умножая на \((1+|x|)^l\) и беря супремум по \(x\), получаем
\[
\|\varphi+\psi\|_l
\le\|\varphi\|_l+\|\psi\|_l.
\]

Таким образом, \(\|\cdot\|_l\) удовлетворяет всем трём аксиомам нормы, что и требовалось.
\end{proof}
\begin{note}
    Таким образом $S(\R^n)$ -- счётнонормированное пространство.
\end{note}
\begin{theorem}
    $S(\R^n)$ -- метризуемое пространство, то есть существует метрика $d$ такая, что \[d(\phi_m, \phi) \ra 0 \Longleftrightarrow \phi_m \xrightarrow{S} \phi, m \ra +\infty.\]
\end{theorem}
\begin{proof}
    Пусть $\phi, \psi \in S(\R^n)$.
    Возьмём метрику
    \[
        d(\phi, \psi) = \sum\limits_{l = 0}^{+\infty} 2^{-l} \dfrac{\|\phi - \psi\|_l}{1 + \|\phi - \psi\|_l}.
    \]
    Пусть $\phi_m \xrightarrow{d} \phi, m \ra +\infty$.
    Но, тогда, в силу неотрицательности слагаемых:
    \[
        \forall l \in \N_{0} \hookrightarrow \dfrac{\|\phi_m - \phi\|_l}{1 + \|\phi_m - \phi\|_l} \ra 0, m \ra +\infty.
    \]
    Рассмотрим $t \to \dfrac{t}{1 + t} = 1 - \dfrac{1}{1 + t}$ -- возрастающая и непрерывная в нуле (и у неё есть обратная функция),
    тогда $\|\phi_m - \phi\|_l \ra 0, m \ra +\infty$. \\
    Покажем теперь в обратную сторону.
    Пусть $\forall l \in \N_0 \hookrightarrow \|\phi - \phi_m\|_l \ra 0, m \ra +\infty$ и надо показать сходимость в метрике.
    Зафиксируем $\epsilon > 0$.
    Найдём $l(\epsilon)$ такое, что
    \[
        \sum\limits_{l(\epsilon)}^{+\infty} 2^{-l} \dfrac{\|\phi_m - \phi\|_l}{1 + \|\phi_m - \phi_m\|} < \epsilon/2.
    \]
    Оно существует, поскольку каждое из слагаемых не больше $1$.
    А теперь выберем $M(\epsilon) \in \N$ таким большим, чтобы
    \[
        \max\limits_{1 \leq l \leq l(\epsilon)} \|\phi - \phi_m\|_l < \epsilon/2 \forall m > M(\epsilon).
    \]
    Но, тогда, разобьём метрику на две части:
    \[
        \sum\limits_{l = 1}^{+\infty} \ldots = \sum\limits_{1}^{l(\epsilon)} \ldots + \sum\limits_{l(\epsilon) + 1}^{+\infty} \ldots \leq \epsilon/2 \cdot\sum\limits_{l = 1}^{l(\epsilon)} 2^{-l} + \epsilon/2 < \epsilon.
    \]
    Что и требовалось показать.
\end{proof}


\begin{note}
    Надо проверить, что $d$ -- действительно метрика.
\end{note}

\begin{note}[Примечание техающего]
Было в качестве упражнения, попробовал доказать сам.
\end{note}
\begin{proof}
Проверим три аксиомы метрики.

\medskip
\noindent 1. Неотрицательность и невырожденность.
Каждое слагаемое 
\[
2^{-l}\,\frac{\|\phi-\psi\|_l}{1+\|\phi-\psi\|_l}
\]
неотрицательно, следовательно $d(\phi,\psi)\ge0$. Более того, 
\[
d(\phi,\psi)=0
\;\Longrightarrow\;
\forall l,\;\frac{\|\phi-\psi\|_l}{1+\|\phi-\psi\|_l}=0
\;\Longrightarrow\;
\forall l,\;\|\phi-\psi\|_l=0
\;\Longrightarrow\;
\phi=\psi.
\]

\medskip
\noindent 2. Симметричность.
Поскольку $\|\phi-\psi\|_l=\|\psi-\phi\|_l$, очевидно
\[
d(\phi,\psi)=d(\psi,\phi).
\]

\medskip
\noindent 3. Неравенство треугольника.
Обозначим для краткости 
\[
f(t)=\frac{t}{1+t},\qquad t\ge0.
\]
Функция $f$ возрастает и удовлетворяет неравенству
\[
f(a+b)\;\le\;f(a)+f(b),
\quad a,b\ge0.
\]
Действительно, используя
\(\|\,\phi-\psi\|_l\le\|\phi-\chi\|_l+\|\chi-\psi\|_l\) и возрастание $f$, получаем
\[
f\bigl(\|\phi-\psi\|_l\bigr)
\;\le\;
f\bigl(\|\phi-\chi\|_l+\|\chi-\psi\|_l\bigr)
\;\le\;
f\bigl(\|\phi-\chi\|_l\bigr)
+f\bigl(\|\chi-\psi\|_l\bigr).
\]
Домножая на $2^{-l}$ и суммируя по $l\ge0$, заключаем
\[
d(\phi,\psi)
=\sum_{l=0}^\infty2^{-l}f\bigl(\|\phi-\psi\|_l\bigr)
\le
\sum_{l=0}^\infty2^{-l}\Bigl[f\bigl(\|\phi-\chi\|_l\bigr)
+f\bigl(\|\chi-\psi\|_l\bigr)\Bigr]
=d(\phi,\chi)+d(\chi,\psi).
\]

Таким образом, $d$ удовлетворяет всем аксиомам метрики.
\end{proof}
\begin{theorem}
    Преобразование Фурье осуществляет линейный изоморфизм $S(\R)$ на $S(\R)$.
    Более того, $\forall m, n \in \N_0$, справедливы следующие равенства:
    \[
        F\left[\dfrac{d^m \phi}{dx^m}\right](y) = (iy)^m F[\phi](y), \quad \dfrac{d^n F[\phi]}{dy^n}(y) = F[(-ix)^n \phi](y).
    \]
\end{theorem}
\begin{proof}
    Пусть $\phi \in S(\R)$, то, по определению:
    \[
        \sup ((1 + |x|)^l \max\limits_{0 \leq k \leq l} |\phi^{(k)}(x)|) = \|\phi\|_l < +\infty \forall l \in \N_0.
    \]
    В частности $\forall k \in \N_0 \hookrightarrow \sup\limits_{x \in \R} |x|^k|\phi (x)| < +\infty$.
    Так как
    \[
        |x|^k|\phi(x)| \leq \dfrac{(1 + |x|)^2}{(1 + |x|)^2} |x|^k|\phi(x)| \leq \dfrac{1}{1 + |x|^2} \|\phi\|_{k + 2}.
    \]
    То и $|x|^k\phi \in L_1(\R) \;\; \forall k \in \N_0$, а значит $F[\phi]$ -- бесконечно-дифференцируемая функция и справедлива вторая формула в утверждении теоремы. \\
    Так как $\forall m \in \N_0 \hookrightarrow \phi^{(m)} \in L_1(\R)$, то
    \[
        F\left[\dfrac{d^m \phi}{dx^m}\right] \ra 0, x \ra \infty.
    \]
    И тогда $\forall m \in \N_0 \hookrightarrow F[\phi](y) = o\left(\frac{1}{|y|^m}\right), y \ra +\infty$. \\
    А значит
    \[
        (iy)^m \dfrac{d^n F[\phi]}{dy^n} = (iy)^m F[(-ix)^n \phi] = F\left[\dfrac{d^m}{dx^m}((-ix)^n \phi)\right].
    \]
    Теперь осталось показать только
    \[
        (*) = \dfrac{d^m}{dx^m}\left((-ix)^n \phi\right) \in L_1(\R).
    \]
    Но это действительно правда, поскольку $\phi$ -- быстро убывающая функция на бесконечности. По правилу Лейбница раскроем производную:
    \begin{multline*}
        \left|\dfrac{d^m}{dx^m}((-ix)^n \phi)\right| = \left|\sum\limits_{k = 1}^m \binom{m}{k}((-ix)^n)^{(k)} \phi^{(m - k)}\right| \leq C (1 + |x|)^{\max(n, m)}\max\limits_{0 \leq s \leq \max(n,m)} |\phi^{(s)}| = C \|\phi\|_{\max(n,m)}.
    \end{multline*}
    А $C \|\phi\|_{\max(n,m)}$ - конечна. Теперь домножим и разделим на $(1 + |x|)^2$
    \[
        (*) = \frac{(1 + |x|)^2}{(1 + |x|)^2} \left|\sum\limits_{k = 1}^m \binom{m}{k}((-ix)^n)^{(k)} \phi^{(m - k)}\right| \leq \frac{(1 + |x|)^2}{(1 + |x|)^2} \left(C (1 + |x|)^{\max(n, m)}\max\limits_{0 \leq s \leq \max(n,m)} |\phi^{(s)}|\right) \leq.
    \]
    \[
        \leq \frac{1}{(1 + |x|)^2} C \|\phi\|_{\max(n,m)+2}.
    \]
    Получили, что m-ая производная мажорируется $\frac{C}{(1+|x|)^2}$, тк $\phi \in S$. \\ Значит $\forall m, n \in \N_0 \hookrightarrow \dfrac{d^m ((-ix)^n \phi)}{dx^m} \in L_1(\R)$ и
    \[
        (iy)^m \dfrac{d^n F[\phi](y)}{dy^n} \ra 0, y \ra \infty.
    \]
    И тогда $F[\phi] \in S(\R)$, а значит $F[\cdot]$ -- действительно преобразование $S(\R)$. \\
    Покажем, что это изоморфизм.
    Поскольку $\phi$ -- бесконечно-дифференцируема и интегрирума, то справедлива формула обращения:
    \[
        F^{-1}\left[F[\phi](x)\right] = F[F^{-1}[\phi]](x) = \phi(x) \quad \forall x \in \R.
    \]
    А значит, если $F[\phi] = 0$, то $\phi \equiv 0$ и $F$ -- инъективное линейное отображение из $S(\R)$ в $S(\R)$. \\
    Вышеприведённые рассуждание верны и для обратного преобразования Фурье, а значит обратное преобразование тоже инъективное линейное отображение, а следовательно $F$ -- изоморфизм.
\end{proof}
\begin{note}
    Эту теоремку бы тоже слегка дописать.
    Или даже не слегка...
\end{note}
\begin{theorem}
    Преобразование Фурье сохраняет сходимость в $S(\R)$, то есть если
    \[
        \phi_m \xrightarrow{S} \phi, m \ra +\infty.
    \]
    То и
    \[
        F[\phi_m] \xrightarrow{S} F[\phi], m \ra +\infty.
    \]
\end{theorem}
\begin{proof}
    Зафиксируем $l \in \N_0$.
    Рассмотрим $l$-ую норму $F[\phi_m - \phi]$:
    \[
        \|F[\phi_m] - F[\phi]\|_{l} = \|F[\phi_m - \phi]\|_{l} = \sup \limits_x (1 + |x|)^l \max\limits_{0 \leq k \leq l} \dfrac{d^k}{dy^k}(F[\phi_m - \phi]) = \sup ((1 + |x|)^l \max\limits_{0 \leq k \leq l} F[(-ix)^k(\phi_m - \phi)]).
    \]
\end{proof}
\begin{note}
    Доказательство на 14-й лекции будет.
\end{note}
\begin{definition}
    Определеним $S'(\R)$ как пространство всех непрерывных (по отношению к сходимости в $S$) линейных функционалов.
\end{definition}
\begin{note}
    Под непрерывностью $\lambda$ имеется в виду 
    \[ \phi_m \xrightarrow{S} \phi, \;\; m \ra \infty \quad => \quad (\lambda, \phi_m) \ra (\lambda, \phi), \;\; m \ra \infty
    \]
\end{note}
\begin{note}
    Заметим, что $D(\R) \subset S(\R)$ и $S'(\R) \subset D'(\R)$.
\end{note}
\begin{definition}
    Пусть $\lambda \in S'(\R)$.
    Тогда $\forall k \in \N \;\, \forall \phi \in S$ определим $(\lambda^k, \phi) = (-1)^k (\lambda, \phi^{(k)})$.
    Доказательство корректности аналогично доказательству для $D$.
\end{definition}
