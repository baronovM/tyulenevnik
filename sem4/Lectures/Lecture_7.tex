\section{Аппроксимация функций}
\subsection{Аппроксимация в пространствах $L_p$}
Для наших целей понадобится приближать наши функции другими, более понятными.

\begin{definition}
    Функция $f : \R^{n} \rightarrow \R$ называется ступенчатой, если она является линейной комбинацией индикаторов ячеек.
\end{definition}

Теперь докажем, что такое функции приближают по норме $L_p$.

\begin{theorem}
    Пусть множество $E \subseteq \R^n$ измеримо, $f \in L_p(E), $ где $p \in [1, +\infty)$.
    Тогда верно следующее:
    $$\forall \varepsilon > 0 \ \exists \ \text{ступенчатая функция } h_{\varepsilon} : ||f - h_{\varepsilon}||_{L_p(E)} < \varepsilon.$$
\end{theorem}

\underline{Идея доказательства:} как обычно мы доказываем это сначала для простых функций, а позже для всех, сводя к уже доказанному с помощью приближений.

\begin{proof} Разобьем доказательство на шаги:

\begin{enumerate}
    \item Пусть $f = \chi_{G},$ где множество $G$ имеет конечную меру. Тогда из определения верхней меры следует:
    $$\forall \varepsilon > 0 \ \exists \{P_{k}\}_{k = 1}^{\infty} : \ \LL^n(G) + \varepsilon \geq \sum_{k  =1}^{\infty}\LL(P_k).$$
    Теперь из сходимости ряда мер ячеек следует, что можно взять такой большой номер $N:$
    $$\sum_{k = N + 1}^{\infty} \LL^n(P_k) < \varepsilon.$$
    По теореме о дизъюнктном представлении в полукольце существует набор непересекающихся $\{Q_l\}_{l = 1}^{m}: \ P_1 \cup \ldots \cup P_N = \bigsqcup_{l = 1}^mQ_l.$ Обозначим за $A = \bigcup_{i = 1}^{\infty} P_k,  B = P_1 \cup \ldots \cup P_N.$ Тогда  
    $$\chi_{B} = \sum_{l = 1}^{m}\chi_{Q_l}.$$
    Возьмем в качестве приближающей ступенчатой функции $\chi_{B}.$ Осталось доказать, что она приближает с точностью до $\varepsilon$ по норме.
    $$||f - h||_{L_p} = \left( \int_{\R^n} |\chi_G(x) - \chi_B(x)|^p dx\right)^{\frac{1}{p}} \leq$$
    $$\leq \left( \int_{\R^n} |\chi_G(x) - \chi_A(x)|^p dx\right)^{\frac{1}{p}} + \left( \int_{\R^n} |\chi_A(x) - \chi_B(x)|^p dx\right)^{\frac{1}{p}} = $$
    $$= \left( \LL^n(A \setminus G)\right)^{\frac{1}{p}} + \left( \LL^n(A \setminus B) \right)^{\frac{1}{p}} \leq 2\varepsilon^{\frac{1}{p}}.$$ 
    \item Если $f$ -- простая, то есть линейная комбинация индикаторов множеств конечной меры, явно сводится к пункту $1$ с помощью неравенства треугольника.
    \item $f \in L_p(E)$ произвольная, тогда из определения интеграла Лебега можно ее приблизить простой с точностью до $\varepsilon / 2, $а простые мы уже умеем приближать ступенчатыми с точностью до $\varepsilon / 2.$ Осталось применить неравенство треугольника и требуемое будет доказано.
\end{enumerate}
\end{proof}

Теперь мы можем показать непрерывность интеграла Лебега по сдвигу.
\begin{theorem}[Непрерывность по сдвигу]
    Пусть $f \in L_p(\R^n), $ где $p \in [1, +\infty).$ Тогда верно следующее:
    $$||f(x) - f(x - h)||_{L_p} \rightarrow 0, h \rightarrow 0$$
\end{theorem}

\underline{Идея доказательства:} Обозначим за $f_h(x) = (x - h),$ заметим, что в силу неравенства треугольника:
$$||f - f_h|| \leq ||f - g|| + ||g - g_h|| + ||f_h - g_h|| \ \forall g \in L_p(\R^n)$$
Ясно, что можно $g$ можно подобрать, чтобы 1 и 3 слагаемые были маленькими, проблема лишь в том, чтобы уменьшить второе слагаемое.  

\begin{proof}
    Заметим, что для любой функции $g \in L_p(\R^n):$
    $$||f_h - g_h|| = \int_{\R^n}|f(x - h) - g(x - h)| dx = \{\text{выполним замену переменной } t = x - h\}=$$
    $$= \int_{\R^n} |f(t) - g(t)| dt = ||f - g||$$
    Тогда в качестве $g$ возьмем ступенчатую функцию, которая приближает $f.$ Осталось теперь доказать, что $g$ можно приблизить $g_h.$ Из теоремы о дизъюнктном представлении следует, что $g$ можно представить в виде:
    $$g(x) = \sum_{k = 1}^{n}a_k \chi_{Q_k}(x), \ Q_k -\text{ячейка}$$
    $$||g - g_h||_p \leq \sum_{k = 1}^{n} |a_k| \cdot ||\chi_{Q_k} - \chi_{Q_k + h}||$$
    Что стремится к нулю при $h \longrightarrow 0.$
\end{proof}

\subsection{Свёртка функций}

\begin{lemma}
    Пусть $f: \R^n \rightarrow \R$ измеримая функция, тогда отображения
    $$(x, y) \mapsto f(x - y)$$
    $$(x, y) \mapsto f(x + y)$$
    измеримы как отображения $\R^{2n} \to \R$.
\end{lemma}

\begin{proof}
    Докажем первое утверждение, второе доказывается аналогично. Обозначим за $E_c = \{x | \ f(x) > c\},$ оно является измеримым из условия леммы. Теперь рассмотрим следующее линейное отображение:
    $$T: (x, y) \longrightarrow (x - y, y).$$
    Оно обратимо, так как определено обратное отображение $T^{-1}((x, y)) = (x + y, y).$ Осталось лишь заметить, что верно:
    $$\{(x, y) | x - y \in E_c\} = T^{-1}(E_C \times \R^n) = \{(x, y) | \ f(x - y) > c\} .$$
    Отсюда следует требуемое.
\end{proof}

Теперь мы готовы к определению свертки функций и к доказательству корректности этого определения.

\begin{theorem}
    Пусть $f, g \in L_1(\R^n).$ Тогда
    \begin{enumerate}
        \item Для $\LL-$почти всех $x \in \R^n$ корректно определена функция (будет называть ее сверткой) $f*g(x) := \int_{\R^n}f(x - y)g(y) dy.$
        \item $f* g$ измерима в широком смысле.
        \item $f * g \in L_1(\R^n)$
        \item $||f * g||_{L_1} \leq ||f||_{L_1} \cdot ||g||_{L_1}$
    \end{enumerate}
\end{theorem}

\begin{proof}
    Рассмотрим следующую функцию:
    $$H(x, y) = |f(x - y)| \cdot |g(y)|.$$
    Ясно, что это неотрицательная, измеримая функция, тогда по теореме Тонелли:
    $$\iint_{\R^{2n}}H(x, y) dx dy = \int_{\R^n}\left( \int_{\R^n} H(x, y) dy\right) dx = \int_{\R^n}\left( \int_{\R^n} H(x, y) dx\right) dy.$$
    Подробнее остановимся на втором интеграле, внутренний интеграл преобразуется так:
    $$\int_{\R^n} H(x, y) dx = \int_{\R^n} |f(x - y)| \cdot |g(y)| dx = |g(y)| \int_{\R^n}|f(x - y)|dx = |g(y)| \cdot ||f||_{L_1}$$
    Тогда весь интеграл:
    $$ \int_{\R^n}\left( \int_{\R^n} H(x, y) dx\right) dy = ||f||_{L_1} \cdot  \int_{\R^n} |g(y) dy = ||f||_{L_1} \cdot ||g||_{L_1} < + \infty$$
    Теперь применим теорему Фубини для $F(x, y) = f(x - y) g(y),$ так как выше мы показали, что $F(x, y) \in L_1(\R^{2n}).$ Тогда пункты $1, 2$ из нее сразу следуют. Покажем оставшиеся:
    $$\int_{\R^n}|f*g|(x) dx \leq \int_{\R^n}\left( \int_{\R^n} |f(x - y)| \cdot |g(y)| dy \right) dx = ||f||_{L_1} \cdot ||g||_{L_1}$$
\end{proof}

Сформулируем еще одну теорему

\begin{theorem}
    Пусть $p \in [1, +\infty), \ f \in L_p(\R^n), \ g \in L_{p'}(\R^n), $ где $\frac{1}{p} + \frac{1}{p'} = 1.$ Тогда:
    \begin{enumerate}
        \item $f * g(x)$ корректно определена для всех $x \in \R^n.$
        \item $f * g(x)$ равномерно непрерывна на $R^n.$
    \end{enumerate}
\end{theorem}

\begin{proof}
Докажем последовательно
    \begin{enumerate}
        \item По неравенству Гельдера получаем:
        $$|f * g(x)| \leq \int_{R^n}|f(x - y)| \cdot |g(y)| dy \leq ||f||_{L_p} \cdot ||g||_{L_{p
        '}} < + \infty$$
        \item Обозначим за $(f * g)_h(x) = f * g(x - h), f_h(x) = f(x -h).$ Верно равенство:
        $$(f * g)_h(x) - f * g(x) = \int_{R^n} f(x - y -h) g(y)dy - f*g(x) = f_h * g(x) - f*g(x)$$
        Теперь оценим отклонение свертки при сдвиге:
        $$|(f * g)_h(x) - f * g(x)| = |f_h * g(x) - f*g(x)| \leq ||f_h - f| * g(x)| \leq ||f-f_h||_{L_p} \cdot ||g ||_{L_p'}$$
        Теперь по уже доказанному утверждению,  получаем, что правая часть стремится к $0$ и при этом оценка не зависит от $x.$ Таким образом, требуемое доказано.
        \item Осталось рассмотреть случаи, когда одно из $p, p'$ равно $+\infty.$ А именно рассмотрим случай, когда $p = \infty, p' = 1.$ Для этого случая достаточно лишь заметить, что совершенно аналогично доказывается неравенство:
        $$|f * g(x)| \leq ||f||_{L_1} \cdot ||g||_{L_{\infty}}$$
    \end{enumerate}
\end{proof}

\subsection{Аппроксимативная единица}
\begin{definition}
    Пусть $f: \R^n \rightarrow \R$. Носителем функции будем называть множество $\supp \ f = \overline{\{x \in \R^n : f(x) \neq 0 \}}$
\end{definition}

\begin{definition}
    Семейство функций $\{\omega_t\}_{t \in (0, +\infty)}$ называется аппроксимативной единицей, если $\forall t > 0$ выполнено следующее: 
    \begin{enumerate}
        \item $\omega_t(x) \ge 0 \ \forall x \in \R^n$.
        \item $\int\limits_{\R^n} \omega_t(x) dx = 1$.
        \item Выполнено фокусирующее свойство: \[\forall \delta > 0  \ \exists \lim\limits_{t \ra +0} \int\limits_{\R^n \setminus B_\delta(0)} \omega_{t}(x)dx = 0.\]
    \end{enumerate}
\end{definition}
\begin{example}[Соболевская аппроксимативная единица]
    Соболевской <<шапкой>> будем называть функцию
    \[
        \psi(x) = \begin{cases}
                      \exp\biggr({-\dfrac{1}{1 - \|x\|^2}}\biggr), & \|x\| < 1 \\
                      0, & \text{ иначе.}
        \end{cases}
    \]
    Функция $\psi \in C^{\infty}(\R^n)$ и её носитель -- замкнутый единичный шар. \\
    Обозначим $\int_{\R^n} \psi(x)dx = C \neq 0$ ~---~ некоторая константа. \\
    Обозначим за $\omega = \psi/C$.
    Тогда $\omega$ -- неотрицательная, бесконечно гладкая функция с компактным носителем и единичным интегралом по всему пространству. \\
    Положим по определению $\omega_\epsilon(x) = \omega(x/\epsilon)/\epsilon^n$.
    Семейство функций $\{\omega_\epsilon\}_{\epsilon \in (0, +\infty)}$ является аппроксимативной единицей.
\end{example}
\begin{corollary}
    Пусть $f \in L_p(\R^n)$.
    Усреднением $f$ по Соболеву будем называть функцию $f_\epsilon = f * \omega_\epsilon$.
    По предыдущей теореме, $f_\epsilon$ -- всюду корректно определённая равномерно непрерывная функция.
\end{corollary}
\begin{note}
    Далее в курсе будет доказано, что $f_\epsilon \in C^{\infty}$.
\end{note}
\begin{theorem}
    Пусть $p \in [1, +\infty)$ и $f \in L_p(\R^n)$. Тогда
    \[
        \forall \epsilon > 0  \ \exists h \in C^{\infty}_0(\R^n) \hookrightarrow \|f - h\|_p \leq \epsilon.
    \]
\end{theorem}
\begin{proof}
    Зафиксируем $\epsilon > 0$.
    В силу свойств интеграла Лебега
    \[
        \exists R > 0 \hookrightarrow \biggr(\int\limits_{\R^n \setminus B_R(0)}|f|^p(x)dx \biggr)^{1/p} < \epsilon.
    \]
    Рассмотрим $g = \chi_{B_R(0)}f$.
    Тогда при достаточно малых $\delta$ верно, что
    \[
        \|g * \omega_\delta - g\|_p < \epsilon.
    \]
    Поскольку $\supp \omega_\delta \subset \overline{B}_\delta(0)$ и $\supp g * w_\delta \subset \overline{B}_{R + \delta}(0)$. \\
    И тогда, по замечанию, $g * \omega_\delta \in C^{+\infty}_0(\R^n)$. \\
    Осталось показать что $\|g * \omega_\delta - g\|_p \ra 0, \delta \ra +0$.
    По определению свёртки:
    \begin{multline*}
        \biggr(\int_{\R^n} |g * \omega_\delta(x) - g(x)|^pdx \biggr)^{1/p} = \biggr(\int_{\R^n} \biggr|\int_{\R^n} g(x - y)\omega_{\delta}(y)dy - 1 \cdot g(x)\biggr|^p dx \biggr)^{1/p} =
        \\ = \biggr(\int_{\R^n} \biggr|\int_{\R^n} g(x - y)\omega_\delta(y)dy - \int_{\R^n}g(x)w_\delta(y)dy \biggr|^p dx \biggr)^{1/p} = \biggr(\int_{\R^n}\biggr|\int_{\R^n} (g(x - y) - g(y))\omega_{\delta}(y)dy\biggr|^p dx\biggr)^{1/p} \leq \\
        \leq \biggr( \int_{\R^n} \biggr( \int_{\R^n} |g(x - y) - g(x)|w_{\delta}^{1/p}(y)\omega_{\delta}^{1/p'}(y)dy \biggr)^p dx  \biggr)^{1/p} \leq
        \\ \leq \biggr(\int_{\R^n} \biggr(\int_{\R^n} |g(x - y) - g(x)|^p \omega_\delta(y)dy \biggr)dx \biggr)^{1/p} \cdot \biggr(\int_{\R^n} \omega_{\delta}^{p'/p'}(y)dy\biggr)^{p/p'} = \\
        = \biggr(\int_{\R^n} \biggr( \int_{\R^n} |g(x - y) - g(x)|^p \omega_\delta(y)dy \biggr) dx \biggr)^{1/p} \leq \biggr(\int_{\R^n}  \omega_\delta(y) \int_{\R^n}|g(x - y) - g(x)|^p dx dy\biggr)^{1/p}.
    \end{multline*}
    В силу непрерывности интеграла Лебега по сдвигу $\int_{\R^n} |g(x - y) - g(x)|^p dx \ra 0. y \ra 0$, а значит
    \[
        \forall \epsilon > 0 \exists \delta(\epsilon) > 0 \forall \|y\| < \delta(\epsilon) \hookrightarrow \int_{\R^n} |g(x - y) - g(x)|^p dx < \epsilon.
    \]
    Из чего и следует требуемая оценка.
\end{proof}