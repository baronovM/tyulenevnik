\section{Интегралы зависящие от параметра}
Интегралы, зависящие от параметра, бывают собственные и несобственные. Для начала рассмотрим собственные и дадим некоторые неформальные комментарии.

\subsection{Собственные интегралы зависящие от параметра}

\definition Пусть $X = (X, \MM, \mu)$ - пространство с мерой, $Y$ - параметрическое множество, $f : X \times Y \ra \overline{\R}$ функция, такая что $\forall y \in Y$ она будет интегрируемой, то есть $f \in L_1(X)$, тогда введём функцию $J(y) := \int\limits_X f(x, y) d\mu(x) \in \overline{\R}$ - собственный интеграл Лебега, зависящий от параметра. \label{pidp}
\\
Нас будут интересовать следующие вопросы:
\begin{itemize}
    \item Когда можно интегрировать по параметру?
    \item Когда можно переходить к пределу по параметру?
    \item Когда можно дифференцировать по параметру?
\end{itemize}


Мы будем исследовать правила, при которых можно менять порядок интеграла и операции примененной к функции. Начнем с интегрирования.

\subsubsection{Когда можно интегрировать по параметру?}

Ответом на этот вопрос фактически является Теорема Фубини.

\theorem Пусть дополнительно известно, что $Y = (Y, \NN, \nu)$ ~---~ пространство с мерой, а также $f \in L_1\left[ (X \times Y, \MM \otimes \NN, \mu \otimes \nu) \right]$. Тогда $J$ интегрируема на $Y$ и справедлива формула: $$\int\limits_Y J(y)d\nu(y) = \int\limits_Y \left( \int\limits_X f(x,y)d\mu(x) \right) d\nu(y) = \iint\limits_{X \times Y} f(x,y) d \mu\otimes\nu(x, y) = \int\limits_X \left( \int\limits_Y f(x, y) d\nu(y) \right) d\mu(x)$$

\subsubsection{Когда можно переходить к пределу по параметру?}

\theorem Рассмотрим собственный интеграл, зависящий от параметра (опр. \ref{pidp}), и пусть дополнительно известно, что $Y = (Y, d)$ ~---~ метрическое пространство. $y_0$ - предельная точка в $Y$, а также пусть $\exists \widetilde{X} \subset X$, т.ч. $\mu(X \bs \widetilde{X}) = 0$, а также:
\begin{enumerate}
    \item $\forall x \in \widetilde{X} \quad \exists \lim\limits_{y \ra y_0} f(x, y) = \ff(x)$
    \item $\exists g(x) \in L_1(X)$ т.ч. при некотором $\delta > 0 \quad \left| f(x,y)\right| \leq g(x) \quad \forall x \in \widetilde{X}$ и $\forall y \in \mathring{B}_\delta(y_0)$
\end{enumerate}


Тогда $\ff \in L_1(X)$ и можно переставлять местами предел и интеграл.

\begin{proof}
	Т.к. определения предела по Коши и по Гейне эквивалентны в метрических пространствах, нам достаточно показать, что $\forall \{y_n\} \subset Y \bs y_0$ такой, что $\lim_{n \ra \infty} y_n = y_0$, выполнено \[
	\exists \lim\limits_{n \ra \infty} J(y_n) = \int\limits_{X} \ff(x) d\mu(x). \tag{*}
	\label{pidp:lim_heine}
	\]

	Итак, пусть $\{y_n\}$ сходится к $y_0$. Тогда $\exists N \in \N$, т.ч. $y_n \in \mathring{B}_\delta(y_0) \quad \forall n \geq N$.

	Определим последовательность функций $g_k(x) := f(x, y_{k + N})$, тогда по 2 условию из формулировки теоремы верно
	$$\left| g_k(x) \right| \leq g(x) \quad \forall x \in \widetilde{X}.$$

	Также по условию 1 выполнено

	$$\exists \lim\limits_{k \ra \infty}g_k(x) = \ff(x) \quad \forall x \in \widetilde{X}.$$

	Тогда по теореме Лебега о мажорируемой сходимости получим \eqref{pidp:lim_heine}, но так как последовательность была выбрана произвольно, значит доказано для любой.
\end{proof}

\subsubsection{Когда можно дифференцировать по параметру?}

\theorem Рассмотрим собственный интеграл, зависящий от параметра (опр. \ref{pidp}), и пусть дополнительно известно, что $Y = \lceil c, \, d\rfloor$ ~---~ промежуток, точка $y_0 \in \lceil c, \, d\rfloor$, а также пусть $f : X \times Y \ra \overline{\R}$ т.ч. $\exists \widetilde{X} \subset X : \mu(X \bs \widetilde{X}) = 0$ и $\exists f'_y(x, y) \quad \forall x \in \widetilde{X} \quad \forall y \in Y$. Пусть $f'_y$ удовлетворяет условию $(L)$ в окрестности точки $y_0$. Тогда функция $J$ дифференцируема в точке $y_0$ и справедлива формула:
$$\frac{d J}{d y}(y_0) = \frac{d}{d y} \int\limits_X f(x, y) d \mu(x) \Bigg|_{y=y_0} = \int\limits_X f'_y(x, y_0) d \mu(x).$$

\proof Фиксируем $h \in \R$ такой, что $y_0 + h \in \lceil c, \, d \rfloor$. Тогда распишем производную по определению
$$\frac{J(y_0 + h) - J(y_0)}{h} = \int\limits_X \frac{f(x, y_0 + h) - f(x, y_0)}{h} d \mu(x) := \int\limits_X F(x, h) d \mu(x)$$

Заметим, что $\exists \lim\limits_{h \ra 0} F(x, h) \Lra \exists f'_y(x, y_0)$. В свою очередь условие $(L)$ для $f'_y$ означает, что $\exists \delta > 0$ т.ч. $\forall y \in U_\delta(y_0) \cap \lceil c, \, d \rfloor$ выполняется $\left| f'_y \right| \leq g(x)$, где $g \in L_1(X)$. Теперь мы хотим свести нашу задачу к предыдущей теореме. Чтобы переставить предел и интеграл, нам надо проверить условие локальной интегрируемости функции $F(x, h)$ в окрестности нуля. Для этого нужно найти мажоранту. При каждом $x \in \widetilde{X}$ применим к функции $f(x, \cdot)$ теорему Лагранжа о среднем:
$$F(x, h) = f'_y(x, \xi(x, h)) \qquad \xi(x, h) \in (y_0, \, y_0 + h)$$

Также верно, что $\xi(x, h) \in \mathring{U}_\delta(y_0) \cap \lceil c, \, d \rfloor$

Так как $f'_y$ удовлетворяет условию $(L)$, если $|h| < \delta$ и $y_0 + h \in \lceil c, \, d \rfloor$, то
$$\left| F(x, h)\right| = |f'_y(x, h)| \leq g(x)$$

Таким образом, получим, что условие $(L)$ выполнено для функции $F$ в нуле. Значит, применяя предыдущую теорему, получим
$$\exists \lim\limits_{h \ra 0} \frac{J(y_0 + h) - J(y_0)}{h} = \int\limits_X \lim\limits_{h \ra 0} F(x, h) d \mu (x) = \int\limits_X f'_y(x, y) d \mu(x)$$

$\qed$

\subsection{Несобственные интегралы зависящие от параметра}

\definition Пусть $- \infty < a < b \leq + \infty$ и $f \in L_1([a, \, b']) \quad \forall b' < b$. Тогда, если $\exists \lim \limits_{b' \ra b - 0} \int\limits^{b'}_a f(x) dx \in \R$, то говорят, что несобственный интеграл Лебега от $f$ на $[a, \, b)$ сходится и обозначают $\int\limits^{\ra b}_a f(x)dx$

\remark Типичный пример несобственного интеграла Лебега
$$\int\limits^{+\infty}_0 \frac{\sin x}{x} d x$$
Этот интеграл не будет сходится в смысле обычного лебеговского интеграла, но существует несобственный интеграл Лебега, и его значение совпадает с римановским.

\definition Пусть $Y$ - параметрическое множество, $f : [a, \, b) \times Y \ra \overline{\R}$ функция, такая что $\forall y \in Y \quad \exists \int\limits^{\ra b}_a f(t, y) d t$, тогда отображение $J(y) = \int\limits_a^{\ra b} f(t, y) d t$ - несобственный интеграл Лебега , зависящий от параметра.

Нас, в целом, будут интересовать те же вопросы, но ситуация осложняется особенностью на конце.

\subsubsection{Когда можно переходить к пределу по параметру?}

\definition Будем говорить, что $\int\limits^{\ra b}_{a} f(t, y) dt$ сходится равномерно по $y \in Y$, если $\forall y \in Y$ он сходится и при этом $\sup\limits_{y \in Y} \left| \int\limits^{\ra b}_{b'} f(t, y) dt \right| \ra 0, \; b' \ra b - 0$

\theorem Пусть дополнительно известно, что $Y$ ~---~ полное метрическое пространство, а $y_0 \in Y$ - предельная точка. Пусть $f : Y \times [a, \, b) \ra \overline{\R}$, т.ч. почти всюду на $[a, \, b) \quad \exists \lim\limits_{y \ra y_0}f(t, y) =: \ff(t)$. Пусть также выполнены следующие условия:
\begin{enumerate}
    \item $\forall b' \in (a, \, b) \quad \ff \in L_1([a, \, b'])$
    \item $\forall b' \in (a, \, b) \quad \exists\lim\limits_{y \ra y_0} \int\limits^{b'}_a f(t, y) dt = \int\limits^{b'}_a \ff(t) dt$
    \item Пусть $\int\limits_a^{\ra b} f(t, y) dt$ сходится равномерно по параметру $y$.
\end{enumerate}


Тогда $\exists \lim\limits_{y \ra y_0} \int \limits^{\ra b}_a f(t, y) dt = \int\limits^{\ra b}_{a} \ff(t)dt$ (*).

\proof Зафиксируем $\varepsilon > 0$, тогда из условия 2 $\Ra \exists\delta(\varepsilon) > 0$  т.ч. $\forall y \in \mathring{B}_\delta (y_0)$ выполняется
$$(\vee) \qquad \left| \int\limits_a^{b'}f(t, y) dt - \int\limits_a^{b'}\ff(t) dt \right| < \frac{\varepsilon}{4}$$

Заметим, что из условия 3 $\Ra \forall \varepsilon \quad \exists b(\varepsilon) \in (a, \, b)$, т.ч. $\forall y \in Y$ выполнено
$$\left| \int\limits_{b(\varepsilon)}^{\ra b} f(t, y) dt \right| < \frac{\varepsilon}{4}$$

Покажем, что $\exists \lim\limits_{y \ra y_0}\int\limits_a^{\ra b} f(t, y) dt$. Для этого проверим условие Коши: $\forall \varepsilon > 0 \quad \exists \delta(\varepsilon, b(\varepsilon)) \Ra \forall y', y'' \in \mathring{B}_\delta(y_0)$ выполнено
$$\left| \int\limits_a^{\ra b}f(t, y') dt - \int\limits_a^{\ra b} f(t, y'')dt \right| \leq \left| \int\limits_{b(\varepsilon)}^{\ra b} f(t, y')dt \right| + \left| \int\limits_{b(\varepsilon)}^{\ra b} f(t, y'')dt \right| + \left| \int\limits_a^{b(\varepsilon)}f(t, y') dt - \int\limits_a^{b(\varepsilon)} f(t, y'')dt \right|$$

Учитывая результаты полученные выше
$$\left| \int\limits_{b(\varepsilon)}^{\ra b} f(t, y')dt \right| + \left| \int\limits_{b(\varepsilon)}^{\ra b} f(t, y'')dt \right| + \left| \int\limits_a^{b(\varepsilon)}f(t, y') dt - \int\limits_a^{b(\varepsilon)} f(t, y'')dt \right| < \varepsilon$$

Условие Коши выполнено, значит предел существует.

Теперь покажем справедливость (*)

$(\vee) \Ra \forall \varepsilon > 0$ выполняется
$$\left| \int\limits^{b(\varepsilon)}_a \ff(t) dt - \lim\limits_{y \ra y_0} J(y) \right| \leq \left| \int\limits^{b(\varepsilon)}_a \ff(t) dt - \int\limits^{\ra b}_a \ff(t, y) dt \right| + \left| \int\limits^{\ra b}_a \ff(t, y) dt - \lim\limits_{y \ra y_0} J(y) \right| \qquad \forall y \in B_\delta(y_0)$$

Из показанного выше известна оценка $\left| \int\limits^{\ra b}_a \ff(t, y) dt - \lim\limits_{y \ra y_0} J(y) \right| < \varepsilon$. Для первого слагаемого оценка получается следующим образом
$$\left| \int\limits^{b(\varepsilon)}_a \ff(t) dt - \int\limits^{\ra b}_a \ff(t, y) dt \right| \leq \left| \int\limits_a^{b(\varepsilon)}\ff(t, y) dt - \int\limits_a^{b(\varepsilon)}\ff(t) dt\right| + \left| \int\limits_{b(\varepsilon)}^{\ra b} f(t, y) dt \right| < \frac{\varepsilon}{2}$$

Значит в результате получим

$$\left| \int\limits^{b(\varepsilon)}_a \ff(t) dt - \lim\limits_{y \ra y_0} J(y) \right| < 2 \varepsilon$$

Что доказывает требуемое равенство. $\qed$

\begin{comment}

\subsubsection{Когда можно интегрировать по параметру?}

\theorem Пусть дополнительно известно, что $Y = (Y, \NN, \nu)$ - пространство с мерой, а также функция $f: [a, \, b) \times Y \ra \overline{\R}$, такая что 
\begin{enumerate}
    \item $\nu(Y) < + \infty$
    \item $\forall b' \in (a, \, b)$ выполнено $f \in L_1([a, \, b'] \times Y, \LL^1 \otimes \nu)$
    \item $\forall y \in Y$ сходится $\int\limits^{\ra b}_a f(t, y) dt$
    \item интеграл $\int\limits^{\ra b}_{a} f(t, y) dt$ сходится равномерно по параметру $y \in Y$.
\end{enumerate}


Тогда $J \in L_1(Y)$ и справедливо выражение
$$\int\limits_Y\int\limits^{\ra b}_a f(t, y)dtd\nu(y) = \int\limits_a^{\ra b} \int\limits_Y f(t, y) d\nu(y) dt$$

\proof Заметим, что $J(y) = \lim\limits_{b' \ra b - 0}\int\limits_a^{b'} f(t, y) dt =: \lim\limits_{b' \ra b - 0} F(b', y)$. По теореме Фубини $\forall$ фиксированного $b'$ верно, что $\exists \int\limits_Y F(b', y) d\nu(y)$. С другой стороны, $J$ - равномерный предел семейства функций $\{ F(b', \cdot) \}_{b' \in (a, \, b)}$. Значит, т.к. $\nu(Y) < +\infty$, получается, что
$$\int\limits_Y \left| J(y) \right| d\nu(y) \leq \int\limits_Y\left| J(y) - \int\limits_a^{b'} f(t, y) dt \right| d\nu(y) + \int\limits_Y \left| \int\limits_a^{b'} f(t, y) dt \right| d\nu(y)$$

Но, т.к. $\left| J(y) - \int\limits_a^{b'} f(t, y) dt \right| < 1$ и $\int\limits_Y \left| \int\limits_a^{b'} f(t, y) dt \right| d\nu(y) < +\infty$ получим
$$\int\limits_Y\left| J(y) - \int\limits_a^{b'} f(t, y) dt \right| d\nu(y) + \int\limits_Y \left| \int\limits_a^{b'} f(t, y) dt \right| d\nu(y) < \nu(Y) + \int\limits_a^{b'} \left| f(t, y) \right| dt d\nu(y) < +\infty$$

Значит $J \in L_1(Y)$, осталось доказать справедливость равенства. (лекция закончилась)

$\qed$

\end{comment}