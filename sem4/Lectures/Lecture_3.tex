\section{Сходимость ряда Фурье в точке.}

Мы начнём с формулировки общего критерия сходимости, не требующего знания конкретных свойств регулярности функций. Поэтому мы называем его ``абстрактным''. 
Он бесполезен с практической точки зрения, поскольку по сути является переформулировкой определения. С другой стороны, такая формулировка окажется полезной при доказательстве конкретных
признаков сходимости рядов Фурье. 


\subsection{Абстрактный критерий сходимости.}

Комбинируя \eqref{eqq.6}, \eqref{eqq.7}, и пользуясь четностью ядра Дирихле, при $f \in L_{1}([-\pi,\pi])$ получим
\begin{equation}
	\label{eqq.19}
	\begin{split}
		&S_{n}[f](x):=\frac{1}{2\pi}\int\limits_{-\pi}^{\pi}\frac{\sin(n+\frac{1}{2})(t-x)}{\sin(\frac{(t-x)}{2})}f(t)\,dt \\
		&=\frac{1}{2\pi}\int\limits_{-\pi}^{\pi}\frac{\sin(n+\frac{1}{2})u}{\sin(\frac{u}{2})}f(x-u)\,du =
		\frac{1}{2\pi}\int\limits_{-\pi}^{\pi}\frac{\sin(n+\frac{1}{2})u}{\sin(\frac{u}{2})}f(x+u)\,du.
	\end{split}
\end{equation}

\begin{lemma}
	\label{Lm.convenient_form}
	При $n \in \mathbb{N}$ и $\forall u \in (-\pi,\pi)$ справедливо равенство
	\begin{equation}
		D_{n}(u) = \frac{\sin(nu)}{\pi u} + \frac{1}{2\pi}(\cos(nu)+g(u)\sin(nu)),
	\end{equation}
	где функция $g:(-\pi,\pi) \to \mathbb{R}$ не зависит от $n$ и ограничена на интервале $(-\pi,\pi)$.
\end{lemma}
\begin{proof}
    Используя формулу синуса суммы, получим
\begin{equation}
	D_{n}(u)=\frac{\sin(nu)\cos(\frac{u}{2})}{2\pi\sin(\frac{u}{2})} + \frac{\cos(nu)}{2\pi} = \frac{\sin(nu)}{\pi u}+\frac{1}{2\pi}(\cos(nu)+g(u)\sin(nu)),
\end{equation}
где мы положили $g(0)=0$ и
$$
g(u):=\frac{1}{\tg(\frac{u}{2})}-\frac{2}{u}, \quad u \in (-\pi,\pi) \setminus \{0\}.
$$
Нетрудно видеть, что $g$  -- нечетная на $(-\pi,\pi)$ и монотонно убывает. Поэтому, $|g(u)| \le 2/\pi$ при всех $u \in (-\pi,\pi)$.
\end{proof}


\begin{theorem}[ ``абстрактный'' критерий сходимости ряда Фурье в точке]
	\label{Th.abstract_criterion}
	Пусть $f:\mathbb{R} \to \mathbb{R}$ является $2\pi$-периодичной, и $f \in L_{1}([-\pi,\pi])$.
	Ряд Фурье $f$ сходится в точке $x \in \mathbb{R}$ к числу $S \in \mathbb{R}$ в том и только том случае, если существует $\delta \in (0,\pi]$ такое, что
	\begin{equation}
		\label{eqq.22}
		\lim\limits_{n \to \infty}\int\limits_{0}^{\delta}\Bigl[\frac{f(x+u)+f(x-u)}{2}-S\Bigr]\frac{\sin(nu)}{u}\,du = 0.
	\end{equation}
\end{theorem}
\begin{proof} Нам будет удобно сделать несколько шагов.

\textit{Шаг 1.}
В силу леммы \ref{Lm.convenient_form}  мы можем переписать \eqref{eqq.19} в виде
\begin{equation}
	\label{eqq.23}
	\begin{split}
		&S_{n}[f](x)=\int\limits_{-\pi}^{\pi}\frac{\sin(nu)}{\pi u}f(x+u)\,du + \varepsilon_{n}[f](x) = \int\limits_{-\pi}^{\pi}\frac{\sin(nu)}{\pi u}f(x-u)\,du + \varepsilon_{n}[f](x)\\
		&=\int\limits_{-\pi}^{\pi}\Bigl[\frac{f(x-u)+f(x+u)}{2}\Bigr]\frac{\sin(nu)}{\pi u}\,du + \varepsilon_{n}[f](x),
	\end{split}
\end{equation}
где мы положили (равенство справедливо в силу четности косинуса и в силу четности произведения $g(u)\sin(nu)$)
\begin{equation}
	\label{eqq.24}
	\begin{split}
		&\varepsilon_{n}[f](x):=\frac{1}{2\pi}\int\limits_{-\pi}^{\pi}f(x+u)(\cos(nu)+g(u)\sin(nu))\,du\\
		&=\frac{1}{2\pi}\int\limits_{-\pi}^{\pi}f(x-u)(\cos(nu)+g(u)\sin(nu))\,du.
	\end{split}
\end{equation}
По теореме Римана--Лебега \ref{Th.Riemann_Lebesgue} имеем $\varepsilon_{n}[f](x) \to 0$, $n \to \infty$.

\textit{Шаг 2.} Используя рассуждения предыдущего шага для $f \equiv 1$, получим
\begin{equation}
	\label{eqq.25}
	\int\limits_{-\pi}^{\pi}\frac{\sin(nu)}{\pi u}\,du = 1 + o(1), \quad n \to \infty.
\end{equation}

\textit{Шаг 3.} Комбинируя \eqref{eqq.22} и \eqref{eqq.25}, имеем
\begin{equation}
	\label{eqq.26}
	|S-S_{n}[f](x)| = \int\limits_{-\pi}^{\pi}\Bigl[\frac{f(x+u)+f(x-u)}{2}-S\Bigr]\frac{\sin(nu)}{\pi u}\,du + o(1), \quad n \to \infty.
\end{equation}

\textit{Шаг 4.}  Функция $1/u$ ограничена на интервале $(\delta,\pi)$ (при любом фиксированном $\delta > 0$). Следовательно, по теореме Римана--Лебега получим
\begin{equation}
	\notag
	\int\limits_{\delta}^{\pi}\Bigl[\frac{f(x+u)+f(x-u)}{2}-S\Bigr]\frac{\sin(nu)}{\pi u}\,du = o(1), \quad n \to \infty.
\end{equation}
Следовательно, учитывая четность функции $\sin(nu)/u$, получим
\begin{equation}
	\label{eqq.27}
	\begin{split}
		&\int\limits_{-\pi}^{\pi}\Bigl[\frac{f(x+u)+f(x-u)}{2}-S\Bigr]\frac{\sin(nu)}{\pi u}\,du = 2\int\limits_{0}^{\delta}\Bigl[\frac{f(x+u)+f(x-u)}{2}-S\Bigr]\frac{\sin(nu)}{\pi u}\,du\\ 
		& + 2\int\limits_{\delta}^{\pi}\Bigl[\frac{f(x+u)+f(x-u)}{2}-S\Bigr]\frac{\sin(nu)}{\pi u}\,du\\ 
		&= 2\int\limits_{0}^{\delta}\Bigl[\frac{f(x+u)+f(x-u)}{2}-S\Bigr]\frac{\sin(nu)}{\pi u}\,du + o(1), \quad n \to \infty.
	\end{split}
\end{equation}

Комбинируя \eqref{eqq.26} и \eqref{eqq.27}, получим \eqref{eqq.22}.

Теорема доказана.
\end{proof}

\begin{corollary}[Принцип локализации для рядов Фурье]
	Пусть даны $2\pi$-периодические функции $f, g \in L_1([-\pi, \pi])$. Тогда, если для некоторой точки $x$ $\exists \delta > 0$ такой, что для $f|_{(x - \delta, x + \delta)} \equiv g|_{(x - \delta, x + \delta)}$, то ряды Фурье этих функций в точке $x$ сходятся и расходятся одновременно.
\end{corollary}
\subsection{Признаки поточечной сходимости рядов Фурье.}
\begin{theorem}[Признак Дини]
	Пусть дана $2\pi$-периодическая функция $f \in L_1([-\pi, \pi])$ и для некоторой точки $x$ существуют $\delta \in (0, \pi)$ и $S \in \R$ такие, что \[
																																							\int_0^\delta |f(x + u) + f(x - u) - 2S|\dfrac{du}{u} < +\infty.
	\]
	то ряд Фурье функции $f$ сходится в точке $x$ к $S$.
\end{theorem}
\begin{proof}
	В силу критерия сводимости ряда Фурье в точке, необходимо и достаточно показать \[
																						\lim\limits_{n \rightarrow +\infty} \int_0^\delta \biggr(\dfrac{f(x + u) + f(x - u)}{2} - S\biggr)\dfrac{\sin nu}{u}du = 0.
	\]
	Поскольку \begin{multline*}
				  \biggr|\int_0^\delta \biggr(\dfrac{f(x + u) + f(x - u)}{2} - S\biggr)\dfrac{\sin nu}{u}du\biggr| \leq \\ \leq \dfrac{1}{2}\int_0^\delta |f(x + u) + f(x - u) - 2S|\dfrac{|\sin nu|}{u}du \leq \\ \leq \dfrac{1}{2}\int_0^\delta \dfrac{|f(x + u) + f(x - u) - 2S|}{u}du < +\infty.
	\end{multline*}
	То если мы зафиксируем $\epsilon > 0$ в силу абсолютной непрерывности интеграла Лебега найдётся $\delta(\epsilon)$ такой, что \[
																																	  \dfrac{1}{2}\int_0^\delta |f(x + u) + f(x - u) - 2S|\dfrac{du}{u} < \dfrac{\epsilon}{2}.
	\]
	А значит по теореме Римана-Лебега \[
										  \int_{\delta(\epsilon)}^\delta \dfrac{f(x + u) + f(x - u) - 2S}{u}\sin nu\cdot du \rightarrow 0, n \rightarrow +\infty.
	\]
	Поскольку данные оценки верны для любого $\epsilon$, то теорема доказана.
\end{proof}

\begin{theorem}[Признак Дирихле--Жордана]
	Если функция $f:\mathbb{R} \to \mathbb{R}$ $2\pi$-периодична, и $f \in BV((a,b)) \cap L_{1}([-\pi,\pi])$ для некоторого интервала $(a,b)$, то её ряд Фурье сходится в каждой точке $x \in (a,b)$, причем
	$$
	S_{n}[f](x) \to \frac{f(x+0)+f(x-0)}{2}, \quad n \to \infty.
	$$
	% Если дополнительно потребовать, что $f \in C((a,b))$, то ряд Фурье функции $f$ сходится к ней равномерно на любом отрезке $[a',b'] \subset (a,b)$.
\end{theorem}

\textit{Доказательство.}
В силу теоремы о представлении функции ограниченной вариации в виде разности двух неубывающих функций, достаточно рассмотреть случай, когда $f$ не убывает.

\textit{Шаг 1.} Зафиксируем точку $x_{0} \in (a,b)$.
В силу теоремы \ref{Th.abstract_criterion} достаточно доказать, что при некотором $\delta > 0$
\begin{equation}
	\notag
	I_{n}:=\int\limits_{0}^{\delta}\frac{\sin(nu)}{u}\bigg[f(x_{0}+u)-f(x+0)\bigg]\,du \to 0, \quad n \to \infty.
\end{equation}


\textit{Шаг 2.} По признаку Дирихле несобственный интеграл (понимаемый в смысле Римана или в смысле Лебега)
$$
J:=\int\limits_{-\infty}^{+\infty}\frac{\sin x}{x}\,dx
$$
является сходящимся.
Поэтому существует постоянная $C > 0$ такая, что
\begin{equation}
	\label{eq.int_estimate}
	\Bigl|\int\limits_{t_{1}}^{t_{2}}\frac{\sin(nu)}{u}\,du\Bigr| = \Bigl|\int\limits_{nt_{1}}^{nt_{2}}\frac{\sin(t)}{t}\,dt\Bigr| \le C \quad \forall t_{1} < t_{2}.
\end{equation}

\textit{Шаг 3.} Фиксируем $\varepsilon > 0$ и выберем $\delta(\varepsilon) > 0$ столь малым, что $U_{\delta}(x_{0}) \subset (a,b)$ и при этом $|f(x_{0}+0)-f(x_{0}+u)| < \frac{\varepsilon}{2C}$
при всех $u \in (0,\delta(\varepsilon))$.
В силу второй теоремы о среднем имеем
\begin{equation}
	\notag
	I^{1}_{n}:=\int\limits_{0}^{\delta(\varepsilon)}\frac{\sin(nu)}{u}[f(x_{0}+u)-f(x+0)]\,du =
	[f(x_{0}+\xi)-f(x+0)]\int\limits_{\xi}^{\delta(\varepsilon)}\frac{\sin(nu)}{u}\,du.
\end{equation}
Отсюда и из \eqref{eq.int_estimate} имеем $|I_{n}^{1}| < \varepsilon/2$.


\textit{Шаг 4.} 
В силу теоремы Римана--Лебега \ref{Th.Riemann_Lebesgue} имеем существование такого числа $N_{\varepsilon}:=N(\delta(\varepsilon)) \in \mathbb{N}$, что при $n \geq N_{\varepsilon}$
\begin{equation}
	\notag
	|I_{n}^{2}|:=\Bigl|\int\limits_{\delta(\varepsilon)}^{\delta}\frac{\sin(nu)}{u}[f(x_{0}+u)-f(x+0)]\,du\Bigr| < \varepsilon/2.
\end{equation}

\textit{Шаг 5.} Собирая вышеприведенные оценки, получаем, что $|I_{n}| < \varepsilon$ при всех $n \geq N_{\varepsilon}$.

Теорема полностью доказана.

\begin{note}
    Признаки Дини и Дирихле Жордана не сравнимы.
    \begin{example}
        
\[
f(x) = 
\begin{cases}
\displaystyle \frac{1}{\left| \ln \frac{x}{2\pi} \right|}, & 0 < x \leq \pi, \\
0, & x = 0, \\
f(-x), & -\pi \leq x < 0.
\end{cases}
\]

        \begin{tikzpicture}
\begin{axis}[
    axis lines = middle,
    xmin = -3.5, xmax = 3.5,
    ymin = 0, ymax = 2,
    samples = 300,
    domain = -3.14:3.14,
    xtick={-3.14, -1.57, 0, 1.57, 3.14},
    xticklabels={$-\pi$, $-\frac{\pi}{2}$, $0$, $\frac{\pi}{2}$, $\pi$},
    ytick={1},
    xlabel = $x$, ylabel = $f(x)$,
    every axis plot post/.style={thick}
]
\addplot [
    domain=0.01:3.14, 
    samples=200, 
] {1/abs(ln(x/(2*3.1415)))};

\addplot [
    domain=-3.14:-0.01, 
    samples=200, 
] {1/abs(ln(abs(x)/(2*3.1415)))};

\addplot+[
    only marks,
    mark=*,
] coordinates {(0, 0)};
\end{axis}
\end{tikzpicture}

Понятно, что$
f \in BV([-\pi, \pi]) \cap C([-\pi, \pi]) 
\quad \Rightarrow$ а значит работает признак Дирихле–Жордана.
Но признак Дини не работает:

\[
\int_0^\delta \frac{\left| f(x+t) + f(x-t) - 2f(x) \right|}{t} \, dt
= 2 \int_0^\delta \frac{f(t)}{t} \, dt 
= 2 \int_0^\delta \frac{dt}{|\ln \left( \frac{t}{2\pi} \right)|} = +\infty
\]
        
    \end{example}
    \begin{example}
        \[
\Psi(x) = 
\begin{cases}
\sqrt{x} \cdot \sin\left( \dfrac{1}{x} \right), & x \in (0, \pi], \\
0, & x = 0, \\
\Psi(-x), & x \in [-\pi, 0).
\end{cases}
\]


$\Psi \notin BV((-\delta, \delta)) \quad \forall \delta > 0 
\quad \Rightarrow$
признак Дирихле–Жордана нельзя использовать.

\[
\text{Проверим сходимость:} \quad
2 \int_0^{\delta} \frac{|\Psi(t)|}{t} \, dt 
\leq 2 \int_0^{\delta} \frac{dt}{\sqrt{t}} 
< +\infty
\]
Признак Дини работает

    \end{example}



\end{note}



% Пусть теперь $f \in C((a,b))$ и $[a',b'] \subset (a,b)$. Тогда для любого $\varepsilon > 0$ существует $\delta(\varepsilon) > 0$ такое, что
% $$
% \sup\limits_{x \in [a',b']}\sup\limits_{|u| < \delta(\varepsilon)}|f(x+u)-f(x)| < \varepsilon.
% $$
