\begin{remark}
    Рассуждая аналогично, в случае $\alpha = 1$ можно получить более грубую оценку: 
    $$
    ||S_n[f] - f||_{C([-\pi, \pi])} \le \frac{C\ln^2n}{n}.
    $$
    Но на самом деле можно при условии $f \in LIP(\R)$ справедлива более сильная оценка: 
    $$
    ||S_n[f] - f||_C \le \frac{C \ln n}{n}.
    $$
\end{remark}

\subsection{Теорема Фейера}
\begin{theorem}
    Пусть $f \in C([-\pi, \pi])$ и $f$ ~---~ $2\pi$ - периодична. Тогда $\sigma_n[f] \underset{\R}{\rightrightarrows} f, \ n \rightarrow \infty.$
\end{theorem}
\begin{proof}
    В силу периодичности $\sigma_n[f]$ и $f$ достаточно доказать, что $\sigma_n[f] \underset{[-\pi, \pi]}{\rightrightarrows} f, \ n \rightarrow \infty$. 
    Поскольку $f \in C([-\pi, \pi]])$, то по теореме Кантора она равномерно непрерывна. Значит её модуль непрерывности стремится к нулю: 
    $$
    \omega(\delta) = \sup\limits_{\underset{|x' - x''| < \delta}{x', x'' \in [-2\pi, 2\pi]}} |f(x') - f(x'')| \rightarrow 0, \delta \rightarrow +0.
    $$
    Формально, описанное выше выражение определено для $\delta \in (0, 4\pi).$ Запишем по определению сумму Фейера: 
    $$
    \sigma_n[f](x) = \int\limits_{-\pi}^{\pi}f(x - t) \Phi_n(t) dt = \int\limits_{-\pi}^{\pi}f(t)\Phi_n(x - t) dt \text{ , где } \Phi_n(t) \text{ ~---~ ядро Фейера}.
    $$
    Тогда: 
    $$
    |\sigma_n[f](x) - f(x)| = \left| \int\limits_{-\pi}^{\pi} f(x - t)\Phi_n(t)dt - \int\limits_{-\pi}^{\pi} \Phi_n(t) f(x) dt \right| \le I_n = \int\limits_{-\pi}^{\pi} \Phi_n(t) |f(x - t) - f(t)| dt = I_1(\delta) + I_2(\delta).
    $$
    $$
    I_1(\delta) = \int\limits_{-\delta}^{\delta}\Phi_n(t) |f(x - t) - f(x)| dt \le \omega_{\delta}[f] \int\limits_{-\delta}^{\delta} \Phi_n(t) dt \le \omega_{\delta}[f].
    $$
    В этой оценке мы ограничиваем сверху $|f(x - t) - f(t)|$ через модуль непрерывности, \newline  a $\int\limits_{-\delta}^{\delta} \Phi_n(t) dt \le 1$.
    $$
    I_2(\delta) = \int\limits_{[-\pi, \pi] \setminus [-\delta, \delta]} \Phi_n(t) |f(x - t) - f(x)| dt.
    $$
    Так как $f$ ~---~ непрерывна на $[-2\pi, 2\pi]$, то $\exists \ M > 0$ такое, что $|f(x)| \le M \ \forall x \in [-2\pi, 2\pi]$. Тогда можем оценить $|f(x - t) - f(x)| \le |f(x)| + |f(x - t)| \le 2M$. \newline
    Из вышеприведенного утверждения и того, что $\forall \delta > 0 \sup\limits_{\delta < |u| < \pi} \Phi_n(u) \rightarrow 0, n \rightarrow \infty$ и ограничения, описанного выше, получаем: 
    $$
    I_2(\delta) \le 2M \int\limits_{[-\pi, \pi] \setminus [-\delta, \delta]} \Phi_n(t) dt \le 2M \sup\limits_{\delta \le |t| \le \pi} \Phi_n(t) \rightarrow 0, n \rightarrow \infty.
    $$
    $\forall \epsilon > 0$ найдем $\delta(\epsilon)$ такое, что $I_1(\delta) < \frac{\epsilon}{2}$. Затем, при фиксированном $\delta(\epsilon)$ выберем $N(\epsilon) \in \N$ таким, что $\forall n > N(\epsilon) \ I_2(\delta) \le \frac{\epsilon}{2}$. \newline 
    Итого, получается, $\forall \epsilon > 0 \ \exists N(\delta(\epsilon)) = \tilde{N}(\epsilon)$ такой, что $\forall n > \tilde{N}(\epsilon) \hookrightarrow I_n < \epsilon$.
\end{proof}
\begin{definition}
    Функция
    \[
        T_n(x) = a_0 + \sum_{k=1}^{n}\bigl(a_k\cos(kx) + b_k\sin(kx)\bigr)
    \]
    называется тригонометрическим полиномом степени \(n\), если $|a_n| + |b_n| \neq 0.$
\end{definition}


\begin{corollary}[Первая теорема Вейерштрасса]
    Пусть $f \in C([-\pi, \pi]])$ и $f(-\pi) = f(\pi)$. Тогда, $\forall \epsilon > 0 \ \exists$ тригонометрический полином $T_{\epsilon}$ такой, что $||f - T_{\epsilon}||_{C([-\pi, \pi])} \le \epsilon$.
\end{corollary}

\begin{corollary}[Теорема Вейерштрасса]
    Пусть $-\infty < a < b < \infty$ и $f \in C([a, b])$. Тогда $\forall \epsilon > 0 \ \exists$ полином $P_{\epsilon}[f]$ такой, что $||f - P_{\epsilon}[f]||_{C([a, b])} < \epsilon$.  
\end{corollary}
\begin{proof}
    Для удобства доказательства перенесем отрезок $[a, b]$ в отрезок $[0, \pi]$. Пусть $x \in [a, b]$, а $t \in [0, \pi]$. Обозначим $\varphi(x)$ ~---~ взаимно однозначная функция, преобразующая точку из первого отрезка в точку из второго отрезка. Тогда $x(t) = \varphi^{-1}(t) = a + \frac{b - a}{\pi} t$. \newline
    Заметим, что $f \circ \varphi \in C([0, \pi])$. Продолжим $f$ чётным образом. Получим функцию $\tilde{f} \in C([-\pi, \pi])$ и $\tilde{f}(-\pi) = \tilde{f}(\pi)$. \newline
    Применим теорему Фейера к функции $\tilde{f}$. 
    $$
    \forall \epsilon > 0 \ \exists N(\epsilon) \in \N \ \forall n \ge N(\epsilon) \hookrightarrow \sigma_n[\tilde{f}] \text{ такая, что } ||\tilde{f} - \sigma_n[\tilde{f}]||_{C([-\pi, \pi])} < \epsilon.
    $$
    $$
    \sigma_n[f] = \frac{1}{n} \sum\limits_{k = 1}^{n - 1} S_k[\tilde{f}] \text{ , где } S_k[\tilde{f}] = \frac{a_0}{2} + \sum\limits_{j = 1}^{k} a_j(\tilde{f}) \cos(jx) + \sum\limits_{j = 1}^{k} b_j(\tilde{f}) \sin(jx) 
    $$
    Вспомним, что $\cos(jx)$ и $\sin(jx)$ ~---~ аналитические $\forall j \in \N$. Следовательно, на любом отрезке $[-A, A] \subset \R $ к ним равномерно сходятся их ряды Тейлора. Тогда мы можем приблизить $\cos(jx)$ и $\sin(jx)$ полиномами Тейлора настолько, чтобы после сложения получилось что-то <<небольшое>>. Обозначим $P_{j}(x)$ ~---~ полином Тейлора для $\sin(jx)$, а $Q_{j}(x)$ ~---~ полином Тейлора для $\cos(jx)$. \newline 
    Можно выбрать полиномы Тейлора так, чтобы существовали $\epsilon_{j}$ и $\tilde{\epsilon_j}$ такие, что:
    $$
    \sup\limits_{x \in [-\pi, \pi]} |P_j(x) - \sin(jx)| < \epsilon_{j}
    $$
    $$
    \sup\limits_{x \in [-\pi, \pi]} |Q_j(x) - \cos(jx)| < \tilde{\epsilon_j}
    $$
    И при этом выполнялось:
    $$
    \dfrac{1}{n}\sum\limits_{k = 0}^{n - 1} (\dfrac{a_0}{2} + \sum\limits_{j = 1}^{k}|a_j|\epsilon_j + |b_j|\tilde{\epsilon_j}) < \epsilon
    $$
    Тогда, полагая
    $$
    P_{\epsilon}[\tilde{f}] := \frac{1}{n} \sum\limits_{k = 0}^{n - 1} (\frac{a_0}{2} + \sum\limits_{j = 1}^{k} a_j(\tilde{f}) Q_j + b_j(\tilde{f}) P_j)
    $$
    $P_{\epsilon}[f](t)$ <<живет>> на отрезке $[-\pi, \pi]$. Теперь мы хотим перенести его на $[0, 1]$. \newline
    Положим $t(x) = \frac{x - a}{b - a}\pi$. Тогда $P_{\epsilon}[f] = P_{\epsilon}[\tilde{f}](t(x))$ ~---~ искомый полином, так как $\tilde{f}(t(x)) = f(x)$. Тогда заметим, что $\sup\limits_{t \in [-\pi, \pi]} |\tilde{f}(t) - P_{\epsilon}[\tilde{f}](t)| = \sup\limits_{x \in [a, b]} | f(x) - P_{\epsilon}[f](x)| < \epsilon.$
\end{proof}

\subsection{Скорость убывания коэффициентов Фурье}
Общая концепция: чем более гладкая функция, тем быстрее убывают коэффициенты Фурье. 

\begin{lemma}[Основная]
    Пусть $f \in \tilde{L_1}(\R) \cap BV(\R)$. Тогда $c_f(y) = f(x) e^{-ixy} = O(\frac{1}{n}), y \rightarrow \infty$.
\end{lemma}
\begin{proof}
    Так как $f \in BV(\R)$, то $f(x) = u(x) + v(x), \ x \in \R$, где $u(x)$ ~---~ нестрого возрастающая функция на $\R$, а $v(x)$ ~---~ нестрого убывающая функция на $\R$. \newline
    Тогда можно записать $\forall a, b : -\infty < a < b < \infty$:
    $$
    c_{[a, b]}(y) = \int\limits_{a}^{b} f(x) e^{-ixy} dx = \int\limits_{a}^{b} u(x) e^{-ixy} dx + \int\limits_{a}^{b} v(x) e^{-ixy} dx \Rightarrow
    $$
    $$
    \exists \ \xi \in [a, b], \zeta \in [a, b] : c_{[a, b]}(y) = u(a + 0) \int\limits_{a}^{\xi} e^{-ixy} dx + u(b - 0)\int\limits_{\xi}^{b} e^{-ixy} dx + 
    $$
    $$
     + v(a + 0) \int\limits_{a}^{\zeta} e^{-ixy} dx + v(b - 0) \int\limits_{\zeta}^{b} e^{-ixy} dx.
    $$
    Ключевое наблюдение: если $f \in BV(\R)$ и интрегрируема, то $f(x) \rightarrow 0, x \rightarrow \infty$. \newline
    Пусть $f \nrightarrow 0$. Тогда $\exists C > 0$ такой, что $\forall \delta > 0 \ \exists \ x : |f(x)| > C$. Но при этом, $f \in L_1(\R)$, так как интегрируема. Тогда, $\exists \ \tilde{x} : |\tilde{x}| > \delta \  |f(\tilde{x})| < \frac{C}{2}$. Получаем противоречие, так как можно получить бесконечный набор точек $\{x_n\}$ и $\{\tilde{x}_n\}$, которые мы набираем по описанному выше условию. \newline
    Ограничим: 
    $$
    \left| \int\limits_{a}^{\xi} e^{-ixy} dx \right| < \frac{2}{|y|} \hspace{100pt} \left| \int\limits_{\xi}^{b} e^{-ixy} dx \right| < \frac{2}{|y|}
    $$
    $$
    \left| \int\limits_{a}^{\zeta} e^{-ixy} dx \right| < \frac{2}{|y|} \hspace{100pt} \left| \int\limits_{\zeta}^{b} e^{-ixy} dx \right| < \frac{2}{|y|}
    $$
    $$
    u(a + 0) \le V_{\R}(f) \hspace{102pt} u(b - 0) \le V_{\R}(f)
    $$
    $$
    v(a + 0) \le V_{\R}(f) \hspace{102pt} v(b - 0) \le V_{\R}(f)
    $$
    Получаем: $|c_{[a, b]}(y)| \le \frac{8 V_{\R}(f)}{|y|}$ ~---~ оценка не зависит от выбора интервала $[a, b]$. Устремляя $a \rightarrow -\infty, \ b \rightarrow +\infty$, получаем требуемое. 
\end{proof}

\begin{theorem}[б/д]
    Пусть $F \in AC([a, b])$. Тогда $F$ почти всюду имеет классическую производную и, более того, восстанавливается через свою производную по формуле Ньютона - Лейбница. 
\end{theorem}

\begin{theorem}[Интегрирование по частям, б/д]
    Пусть $F \in AC([a, b]), g \in L_1([a, b])$. Тогда верна формула интегрирования по частям: $\int\limits_{a}^{b} F(x)g(x) dx = F(x)G(x)\big|_{a}^{b} - \int\limits_{a}^{b} F'(x)G(x) dx$, \newline где $G(x) = \int\limits_a^{x} g(t)dt $
\end{theorem}

\begin{corollary}
    Пусть функция $f: \R \rightarrow \R$ ~---~ $2\pi$-периодическая, такая, что $f^{(k - 1)} \in AC([-\pi, \pi])$. Пусть $f^{(k)}$ почти всюду существует и может быть изменена на множестве меры нуль таким образом, что $f^{(k)} \in BV([-\pi, \pi])$. Тогда $c_n(f) = \frac{1}{2\pi} \int\limits_{-\pi}^{\pi} f(x) e^{-inx} dx = O\left(\frac{1}{n^{k + 1}}\right)$.
\end{corollary}

\begin{proof}
    $$
    \int\limits_{-\pi}^{\pi} f'(x) e^{inx} dx = f(x) e^{-inx} \big|_{-\pi}^{\pi} + in \int\limits_{-\pi}^{\pi} f(x) e^{-inx} dx.
    $$
    Проделаем эту операцию $k$ раз. Так как $f$ ~---~ $2\pi$-периодична и $f^{(k)} \in AC([-\pi, \pi])$ ~---~ тоже $2\pi$-периодична: 
    $$
    \int\limits_{-\pi}^{\pi} f^{(k)}e^{-inx} dx = (in)^k \int\limits_{-\pi}^{\pi} f(x)e^{-inx} dx.
    $$
    Но $f^{(k)}$ можно считать $BV([-\pi, \pi])$. \newline
    Рассмотрим функцию $F = 
    \begin{cases}
        f^{(k)}(x), x \in [-\pi, \pi] \\ 
        0, \text{ иначе}
    \end{cases}
    $. Тогда $F \in BV([-\pi, \pi])$ и $\int\limits_{-\pi}^{\pi}f(x) e^{-inx} dx = $ $ \int\limits_{\R} F(x) e^{-inx} dx = O\left(\frac{1}{n}\right), n \rightarrow \infty$ в силу леммы. 
    С учетом того, что $\int\limits_{-\pi}^{\pi} f(x)e^{-inx} dx = \frac{1}{(in)^k}\int\limits_{-\pi}^{\pi} f^{(k)}(x) e^{inx} dx$ получаем требуемое.
    
\end{proof}
