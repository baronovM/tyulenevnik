\section{Консультация}

\subsection{N-свойство Лузина}

\begin{definition}
    Пусть $f: \R^n \rightarrow R$. Будем говорить, что функция обладает N-свойством Лузина, если для любого измеримого множества $E \subset \R^n \ \hookrightarrow \LL^n(E) = 0 \Rightarrow \LL^1(f(E)) = 0$. 
\end{definition}

\begin{theorem}
    Пусть $f \in AC([a, b])$. Тогда она обладает N-свойством Лузина
\end{theorem}

\begin{proof}
    Пусть $E \in [a, b], \LL^1(E) = 0$. В силу регулярности меры Лебега $\forall \delta > 0$ существует открытое множество $E_{\delta} \supset E$ и $\LL^1(E_\delta) < \delta$. 

    \noindent \textbf{Факт:} любое открытое множество $G$ на числовой прямой может быть представлено в виде не более, чем счётного объединения попарно непересекающихся интервалов. 

    \noindent Тогда $\forall \delta > 0 \  E_{\delta} = \bigsqcup\limits_{k = 1}^{\infty} (a_k(\delta), b_k(\delta)) \supset E$. Получаем, что $\sum\limits_{k = 1}^{\infty} |b_k(\delta) - a_k(\delta)| < \delta$.

    \noindent Заметим, что $f([a_k(\delta), b_k(\delta)]) = [c_k(\delta), d_k(\delta)]$ ~---~ образ отрезка есть отрезок. Кроме того, $$f(E) \subset f(E_{\delta}) = \bigcup\limits_{k = 1}^{\infty} f([a_k(\delta), b_k(\delta)]) = \bigcup\limits_{k = 1}^{\infty} [c_k(\delta), d_k(\delta)].$$  

    \noindent Выберем $x_k(\delta) \in [a_k(\delta), b_k(\delta)]$ и $y_k(\delta) \in [a_k(\delta), b_k(\delta)]$ такие, что $f(x_k(\delta)) = c_k(\delta)$ и $f(y_k(\delta)) = d_k(\delta)$. Но тогда интервалы $(x_k(\delta), y_k(\delta))$ ~---~ попарно не пересекаются. Но так как $f \in AC([a, b])$, то $\forall \epsilon > 0 \ \exists \delta(\epsilon)$ такой, что для любой системы из попарно непересекающися интервалов суммарная разница длин которых не более $\delta(\epsilon)$ выполнено, что сумма модулей разности значений меньше $\epsilon$. 

     \noindent $\forall \epsilon > 0 \ \exists$ дизъюнктный набор интервалов $\{(x_k(\delta(\epsilon)), y_k(\delta(\epsilon)))\}_{k = 1}^{\infty}$ и $\sum\limits_{k = 1}^{\infty} \left|x_k(\delta(\epsilon)) - y_k(\delta(\epsilon))\right| < \delta(\epsilon)$. Но в силу абсолютной непрерывности $f$:  $$\sum\limits_{k = 1}^{\infty} |f(x_k(\delta(\epsilon))) - f(y_k(\delta(\epsilon)))| = \sum\limits_{k = 1}^{\infty} |c_k(\delta(\epsilon)) - d_k(\delta(\epsilon))| < \epsilon.$$ Получаем, что $\LL^{1}_{*}(f(E)) < \epsilon$ ~---~ внешняя мера. Но так как мера Лебега полна, а $\epsilon$ выбрано произвольно, то $f(E)$ ~---~ измеримо и $\LL^1(f(E)) = 0$.  
\end{proof}

\subsection{Теорема Банаха - Тарского}

\begin{theorem}[Банаха - Тарского]
$$
f \in AC([a, b]) \Leftrightarrow 
\begin{cases}
    f \in BV([a, b]) \\ 
    f \in C([a, b]) \\ 
    f - \text{обладает N-свойством Лузина}
\end{cases}
$$    
В одну сторону доказательство понятно и было в курсе. В другую сторону доказательство менее тривиальное и в курсе считается не обязательным. 

\noindent Приведем пример функции из $BV([a, b])$ и $C([a, b])$, но при этом не обладают N-свойством Лузина.

\begin{example}[Канторова лестница]
    Построим канторово множество. Будем строить итеративно: на каждой итерации будем брать один из отрезков, полученных на предыдущей итерации, делить его на три равных части и <<выбрасывать>> среднюю. На $k$-ом шаге у нас будет $2^k$ равных отрезков длины $\left(\frac{2}{4} \right)^{k}$. Обозначим объединение этих отрезков $F_k$. Тогда $F = \bigcap\limits_{k = 1}^{\infty} F_k$. Мера этого множества равна 0, так как $(\frac{2}{3})^{k} \rightarrow 0, k \rightarrow \infty$. 

    \noindent Возникает вопрос: а не выкинем ли мы вообще все точки при таком построении множества? Очевидно нет, так как точно останутся точки 0 и 1. Все точки в множестве мы можем пронумеровать в виде двоичных чисел. На $k-ом$ шаге мы даем каждому отрезку номер ~---~ двоичное число длины $k$. 

    \noindent Теперь мы можем построить канторову лестницу. Подробное описание построения есть в книге <<Действительный анализ в задачах>>. Функция также строится итеративно. На каждой итерации мы делим отрезок, полученный на предыдущей итерации на три части. Приведем пример, как строится первая итерация для объяснения алгоритма построения. На первой итерации отрезок $[0, 1]$ делится на 3 части: $\left[0, \frac{1}{3}\right]$, $\left[\frac{1}{3}, \frac{2}{3} \right]$, $\left[\frac{2}{3}, 1 \right]$. Первому отрезку сопоставляем значение 0 (начала разбиваемого отрезка по $y$), второму отрезку сопоставляем значение $\frac{1}{2}$ ~---~ середина, третьему отрезку сопоставляем значение 1.  
    Далее мы берем отрезки, полученные на предыдущей итерации и работаем с ними аналогичным образом, <<сужая>> отрезки и сопоставляемые им множества. Точки, задающие границы отрезки мы соединяем прямыми. 

    \noindent Таким образом, мы получаем последовательность функций $F_n(x) \rightarrow F(x)$. При этом, $F(1) - F(0) = 1$, но $F'(x) = 0$ п.в. Получаем, что канторова лестница непрерывна, ограниченной вариации, но не абсолютно непрерывна.
\end{example}

\end{theorem}