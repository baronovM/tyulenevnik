2\section{Консультация к досрочному экзамену.}

\subsection{N-свойство Лузина}

\begin{definition}
    Пусть $f: \R^n \rightarrow R$. Будем говорить, что функция обладает N-свойством Лузина, если для любого измеримого множества $E \subset \R^n \ \hookrightarrow \LL^n(E) = 0 \Rightarrow \LL^1(f(E)) = 0$. 
\end{definition}

\begin{theorem}
    Пусть $f \in AC([a, b])$. Тогда она обладает N-свойством Лузина
\end{theorem}

\begin{proof}
    Пусть $E \in [a, b], \LL^1(E) = 0$. В силу регулярности меры Лебега $\forall \delta > 0$ существует открытое множество $E_{\delta} \supset E$ и $\LL^1(E_\delta) < \delta$. 

    \noindent \textbf{Факт:} любое открытое множество $G$ на числовой прямой может быть представлено в виде не более, чем счётного объединения попарно непересекающихся интервалов. 

    \noindent Тогда $\forall \delta > 0 \  E_{\delta} = \bigsqcup\limits_{k = 1}^{\infty} (a_k(\delta), b_k(\delta)) \supset E$. Получаем, что $\sum\limits_{k = 1}^{\infty} |b_k(\delta) - a_k(\delta)| < \delta$.

    \noindent Заметим, что $f([a_k(\delta), b_k(\delta)]) = [c_k(\delta), d_k(\delta)]$ ~---~ образ отрезка есть отрезок. Кроме того, $$f(E) \subset f(E_{\delta}) = \bigcup\limits_{k = 1}^{\infty} f([a_k(\delta), b_k(\delta)]) = \bigcup\limits_{k = 1}^{\infty} [c_k(\delta), d_k(\delta)].$$  

    \noindent Выберем $x_k(\delta) \in [a_k(\delta), b_k(\delta)]$ и $y_k(\delta) \in [a_k(\delta), b_k(\delta)]$ такие, что $f(x_k(\delta)) = c_k(\delta)$ и $f(y_k(\delta)) = d_k(\delta)$. Но тогда интервалы $(x_k(\delta), y_k(\delta))$ ~---~ попарно не пересекаются. Но так как $f \in AC([a, b])$, то $\forall \epsilon > 0 \ \exists \delta(\epsilon)$ такой, что для любой системы из попарно непересекающися интервалов суммарная разница длин которых не более $\delta(\epsilon)$ выполнено, что сумма модулей разности значений меньше $\epsilon$. 

     \noindent $\forall \epsilon > 0 \ \exists$ дизъюнктный набор интервалов $\{(x_k(\delta(\epsilon)), y_k(\delta(\epsilon)))\}_{k = 1}^{\infty}$ и $\sum\limits_{k = 1}^{\infty} \left|x_k(\delta(\epsilon)) - y_k(\delta(\epsilon))\right| < \delta(\epsilon)$. Но в силу абсолютной непрерывности $f$:  $$\sum\limits_{k = 1}^{\infty} |f(x_k(\delta(\epsilon))) - f(y_k(\delta(\epsilon)))| = \sum\limits_{k = 1}^{\infty} |c_k(\delta(\epsilon)) - d_k(\delta(\epsilon))| < \epsilon.$$ Получаем, что $\LL^{1}_{*}(f(E)) < \epsilon$ ~---~ внешняя мера. Но так как мера Лебега полна, а $\epsilon$ выбрано произвольно, то $f(E)$ ~---~ измеримо и $\LL^1(f(E)) = 0$.  
\end{proof}

\subsection{Теорема Банаха - Зарецкого}

\begin{theorem}[Банаха - Зарецкого]
$$
f \in AC([a, b]) \Leftrightarrow 
\begin{cases}
    f \in BV([a, b]) \\ 
    f \in C([a, b]) \\ 
    f - \text{обладает N-свойством Лузина}
\end{cases}
$$    
В одну сторону доказательство понятно и было в курсе. В другую сторону доказательство менее тривиальное и в курсе считается не обязательным. 

\noindent Приведем пример функции из $BV([a, b])$ и $C([a, b])$, но при этом не обладают N-свойством Лузина.

\begin{example}[Канторова лестница]
    Построим канторово множество. Будем строить итеративно: на каждой итерации будем брать один из отрезков, полученных на предыдущей итерации, делить его на три равных части и <<выбрасывать>> среднюю. На $k$-ом шаге у нас будет $2^k$ равных отрезков длины $\left(\frac{2}{4} \right)^{k}$. Обозначим объединение этих отрезков $F_k$. Тогда $F = \bigcap\limits_{k = 1}^{\infty} F_k$. Мера этого множества равна 0, так как $(\frac{2}{3})^{k} \rightarrow 0, k \rightarrow \infty$. 

    \noindent Возникает вопрос: а не выкинем ли мы вообще все точки при таком построении множества? Очевидно нет, так как точно останутся точки 0 и 1. Все точки в множестве мы можем пронумеровать в виде двоичных чисел. На $k-ом$ шаге мы даем каждому отрезку номер ~---~ двоичное число длины $k$. 

    \noindent Теперь мы можем построить канторову лестницу. Подробное описание построения есть в книге <<Действительный анализ в задачах>>. Функция также строится итеративно. На каждой итерации мы делим отрезок, полученный на предыдущей итерации на три части. Приведем пример, как строится первая итерация для объяснения алгоритма построения. На первой итерации отрезок $[0, 1]$ делится на 3 части: $\left[0, \frac{1}{3}\right]$, $\left[\frac{1}{3}, \frac{2}{3} \right]$, $\left[\frac{2}{3}, 1 \right]$. Первому отрезку сопоставляем значение 0 (начала разбиваемого отрезка по $y$), второму отрезку сопоставляем значение $\frac{1}{2}$ ~---~ середина, третьему отрезку сопоставляем значение 1.  
    Далее мы берем отрезки, полученные на предыдущей итерации и работаем с ними аналогичным образом, <<сужая>> отрезки и сопоставляемые им множества. Точки, задающие границы отрезки мы соединяем прямыми. 

    \noindent Таким образом, мы получаем последовательность функций $F_n(x) \rightarrow F(x)$. При этом, $F(1) - F(0) = 1$, но $F'(x) = 0$ п.в. Получаем, что канторова лестница непрерывна, ограниченной вариации, но не абсолютно непрерывна.
\end{example}
\end{theorem}

\section{Консультацию к основному экзамену.}
\subsection{Интеграл Фруллани.}
\begin{lemma}
    Пусть $f \in C([0, +\infty))$ и для любого $A > 0 \hookrightarrow$
    \[
        \int_A^{+\infty} \dfrac{f(x)}{x}dx
    \]
    сходится в несобственном смысле.
    Тогда $\forall a, b > 0$ справедлива формула Фруллани:
    \[
        I = \int_0^{+\infty} \frac{f(ax) - f(bx)}{x}dx = f(0)\ln \frac{b}{a}.
    \]
\end{lemma}
\begin{proof}
    Поскольку
    \[
        I = \lim\limits_{\epsilon \ra +0, E \ra +\infty} \int_{\epsilon}^{E} \dfrac{f(ax) - f(bx)}{x}dx.
    \]
    Введём обозначение:
    \[
        I(\epsilon, E) = \int_{\epsilon}^{E} \dfrac{f(ax) - f(bx)}{x}dx.
    \]
    Поскольку
    \[
        I(\epsilon, E) = \int_{\epsilon}^E \dfrac{f(ax)}{x}dx - \int_{\epsilon}^{E} \dfrac{f(bx)}{x}dx = \int_{a\epsilon}^{aE} \dfrac{f(t)}{t}dt - \int_{b\epsilon}^{bE} \dfrac{f(t)}{t}dt = \int_{a\epsilon}^{b\epsilon} \dfrac{f(t)}{t}dt + \int_{bE}^{aE} \dfrac{f(t)}{t}dt = J(\epsilon) + K(E).
    \]
    Ключевой момент заключается в том, что $K(E) \ra 0, E \ra +\infty$ -- это следует из критерия Коши сходимости в применении к несобственному интегралу, данному в условии.
    А $J(\epsilon)$ оценивается как
    \[
        J(\epsilon) = \int_{a\epsilon}^{b\epsilon} \dfrac{f(t) - f(0)}{t}dt + \int_{a \epsilon}^{b\epsilon} \dfrac{f(0)}{t}dt = f(0) \ln \frac{b}{a} + \int_{a\epsilon}^{b\epsilon} \dfrac{f(t) - f(0)}{t}dt \ra f(0) \ln \frac{b}{a}.
    \]
    Второе слагаемое стремится к 0 из непрерывности $f$:
    \[
        \biggr|\int_{a \epsilon}^{b\epsilon} \dfrac{f(0)}{t}dt \biggr| \leq \sup\limits_{t \in [a\epsilon, b \epsilon]}|f(t) - f(0)| \ln \frac{b}{a} \ra 0, \epsilon \ra +0.
    \]
\end{proof}
\begin{example}
    Рассмотрим интеграл $I = \int_0^1 \frac{x^\alpha - x^\beta}{\ln x}dx$.
    Сделаем замену $x = e^{-t}$ и получим
    \[
        I = -\int_0^{+\infty} \dfrac{e^{-\alpha t} - e^{-\beta t}}{t} e^{-t}dt = \ln \dfrac{a + 1}{b + 1}.
    \]
    с использованием формулы Фруллани к функции $f(t) = e^{-t}$.
\end{example}
\subsection{Почленное дифференциорвание рядов Фурье.}
\begin{reminder}
    Функция $f: [a, b] \to \R$ называется кусочно-непрерывно дифференцируемой на $[a, b]$, если существует конечный набор точек $\{x_i\}_{i = 0}^{N} \subset [a, b]$ такой, что
    \[
        a = x_0 < x_1 < \ldots < x_N = b.
    \]
    И $\forall i \in \{0, \ldots\, N - 1\}$ ограничение на $(x_i, x_{i + 1})$ непрерывно-дифференцируемо и существуют односторонние пределы в точках $x_i$ и $x_{i + 1}$ самой функции и её производной.
\end{reminder}
\begin{note}
    Функции не обязательно вещественно-значные. С комплексно-значными работаем аналогично, отдельно с мнимой и вещественной частью.
\end{note}
\begin{lemma}
    Пусть $f, g: \R \to \mathbb{C}$ непрерывные $2\pi$-периодические функции кусочно-непрерывно-дифференцируемые на $[-\pi, \pi]$.
    Тогда
    \[
        \int_0^{2\pi} f g' dx = - \int_0^{2\pi} f' g dx.
    \]
\end{lemma}
\begin{proof}
    Из конечности разбиений на интервалы у нас существует некоторое общее разбиение $\{x_j\}_{j = 0}^m$ для $f$ и $g$, где обе функции на интервалах будут гладкие.
    А значит на $j$-ом подотрезке мы можем проинтегрировать по частям:
    \[
        \int_{x_j}^{x_{j + 1}} f g' dx + \int_{x_j}^{x_{j + 1}} f' g = fg|_{x_j}^{x_{j + 1}}.
    \]
    Просуммируем по всем подинтервалам и получимЖ
    \[
        \sum\limits_{j = 0}^{m - 1} \int_{x_j}^{x_{j + 1}} fg'dx + \int_{x_{j}}^{x_{j + 1}}f' g dx = \sum\limits_{j = 0}^{m - 1} fg|_{x_j}^{x_{j + 1}} = f(x_m)g(x_m) - f(x_0)g(x_0) = 0.
    \]
    (где последний переход получен их периодичности функций).
\end{proof}
\begin{lemma}
    Пусть $f$ -- непрерывная $2\pi$-периодичная функция кусочно-непрерывно-дифференцируемая на $[0, 2\pi]$.
    Тогда $\forall k \in \Z \hookrightarrow c_k(f') = i_k c_k(f)$.
\end{lemma}
\begin{proof}
    Применим предыдущую лемму к функции $g(x) = e^{-i k x}$ и получим:
    \[
        c_k(f') = \frac{1}{2\pi} \int_0^{2\pi} f'(x)e^{-ikx}dx = -\frac{1}{2\pi} \int_0^{2\pi}(-i k)f(x)e^{-ikx}dx = ikc_k(f).
    \]
\end{proof}
\begin{corollary}[Почленное дифференцирование рядов Фурье]
    Пусть $f$ -- $2\pi$-периодическая непрерывная кусочно-непрерывно-дифференцируемая на $[0, 2\pi]$ функция.
    Пусть
    \[
        f(x) = \int_{k = - \infty}^{+\infty} c_k(f) e^{ikx}.
    \]
    Тогда ряд Фурье функции $f'$ будет выражаться как
    \[
        f'(x) \sim \sum\limits_{k \in \Z, k \neq 0} ik c_k e^{ikx}.
    \]
\end{corollary}
\begin{theorem}
    Пусть $f$ -- непрерывная $2\pi$-периодическая кусочно-непрерывно дифференцируемая функция на $[0, 2\pi]$.
    Тогда её ряд Фурье сходится к ней равномерно.
\end{theorem}
\begin{proof}
    Из условий на $f$ и $f'$ следует интегрируемость $f$ и $f'$ в среднеквадратичном, то есть $f \in L_2([0, 2\pi])$ и $f' \in L_2([0, 2\pi])$ (поскольку $f'$ -- измеримая ограниченная функция).
    В силу неравенства Бесселя:
    \[
        \sum\limits_{k \in \Z} |c_k(f')|^2 \leq \frac{1}{2\pi} \int_0^{2\pi} |f'(x)|dx < +\infty.
    \]
    Поскольку при $k \neq 0$
    \[
        |c_k(f)| = \dfrac{|c_k(f')|}{|k|} \leq \frac{1}{2}\biggr(|c_k(f')|^2 + \frac{1}{k^2}\biggr).
    \]
    А значит $\sum\limits_{k \in \Z} |c_k(f)|$ -- сходится и по признаку Вейерштрасса
    \[
        \sum\limits_{k \in \Z} c_k(f)e^{ikx}
    \]
    сходится равномерно на $[0, 2\pi]$ (а следовательно и на всей числовой прямой).
\end{proof}

\subsection{Почленное интегрирование рядов Фурье.}
\begin{theorem}
    Пусть $f: \R \to \mathbb{C}$ является $2\pi$-периодичной кусочно-непрерывной функцией и ей сопоставлен ряд Фурье:
    \[
        f(x) \sim \sum\limits_{k \in \Z} c_k(f) e^{ikx}.
    \]
    Тогда $\forall y \in \R$
    \[
        \int_0^{y}f(x)dx = \sum\limits_{k \in \Z} \int_0^{y} c_k e^{ikx}dx.
    \]
    и ряд сходится равномерно на всей числовой прямой.
\end{theorem}
\begin{proof}
    Рассмотрим функцию $F(x) = \int_0^{x} (f(t) - c_0)dt$.
    Она является $2\pi$-периодической непрерывной кусочно-непрерывно дифференцируемой функцией (в силу теорем для интеграла Римана с переменным верхним пределом).
    В силу предыдущих утверждений ряд Фурье к ней сходится равномерно на $\R$:
    \[
        F(x) = \sum\limits_{k \in Z} c_k(F) e^{ikx} \
    \]
    Более того, поскольку $F'(x) = f(x)$ всюду, кроме, быть может, конечного числа точек то мы получаем следующее соотношение на коэффициенты рядов Фурье:
    \[
        \forall k \neq 0 \hookrightarrow ik c_k(F) = c_k(f).
    \]
    (здесь мы по сути воспользовались утверждением о почленном дифференцировании рядов Фурье).
    Поскольку $F(0) = 0$, то
    \[
        \sum\limits_{k \in \Z} c_k(F) = 0.
    \]
    Но тогда $c_0(F) = -\sum\limits_{k \in \Z, k \neq 0} c_k(F)$.
    Следовательно
    \[
        F(y) = c_0(F) + \sum\limits_{k \neq 0} c_k(F)e^{iky} = \sum\limits_{k \neq 0} c_k(F)(e^{iky} - 1)dx = \sum\limits_{k \neq 0} \dfrac{c_k(f)}{ik}(e^{iky} - 1).
    \]
    Причём это равенство верно $\forall y \in \R$.
    А теперь если мы подставим определение $F(y)$ и перенесём интеграл от $c_0$ в правую часть то мы получим искомое утверждение.
\end{proof}
\begin{corollary}[Теорема о единственности.]
   Если $f$ является $2\pi$-периодической непрерывной функцией и все её коэффициенты Фурье равны нулю, то $f \equiv = 0$.
\end{corollary}

\begin{proof}
    Поскольку $c_0(f) = 0$, то $F(y) = \int_0^{y} f(x)dx$ является $2\pi$-периодической непрерывно-дифференцируемой функцией.
    В тоже время $c_k(F) = \frac{c_k(f)}{ik}$ при $k \neq 0$, а следовательно $c_k(F) = 0$ для всех $k \neq 0$.
    Но $F$ -- непрерывно-дифференцируема и $2\pi$-периодична, а её ряд Фурье к ней равномерно сходится.
    По определению $F(0) = 0$.
    Значит и $c_0(F) = 0$ и тогда $c_k(F) = 0 \  \forall k \in \Z$ и $F \equiv 0$.
    Поэтому и $f = F' = 0$.
\end{proof}

\subsection{Полнота тригонометрической системы.}
\begin{theorem}
    Система тригонометрических полиномов $\{1, \sin nx, \cos nx\}_{n = 1}^{+\infty}$ полна в пространстве $L_2([-\pi, \pi])$.
\end{theorem}
\begin{proof}
    Пусть $f \in L_2([-\pi, \pi])$.
    Зафиксируем $\epsilon > 0$.
    Известно, что пространство $C_0^\infty([-\pi, \pi])$ плотно в $L_p([-\pi, \pi])$ при $p \in [1, +\infty)$.
    Значит существует такая $h_\epsilon \in C_0^\infty([-\pi, \pi])$, что выполнено следующее:
    \[
        \|h_\epsilon - f\|_2 < \frac{\epsilon}{3}.
    \]
    Пусть есть некоторое $\delta \in (0, \pi)$ и функцию $h_\epsilon^\delta$, которая определяется как
    \[
        h_{\epsilon}^{\delta}(x) =
        \begin{cases}
            h_{\epsilon}(x), & |x| < |\pi - \delta|, \\
            \text{ линейно убывает (или возрастает) от 0 в} -\pi \text{ до } h_\epsilon(-\pi + \delta), & x \in [-\pi, -\pi + \delta] \\
            \text{ линейно убывает (или возрастает) от } h_\epsilon(\pi - \delta) \text{ в } \pi - \delta \text{ до } 0 \text{ при } x = \pi, & x \in [\pi - \delta, \pi].
        \end{cases}
    \]
    Такая функция лежит в пространстве $C^*([-\pi, \pi])$ -- непрерывных $2\pi$-периодических функций с равномерной нормой.
    Пусть $M = \max\limits_{\R} |h_\epsilon|$.
    Оценим разность норм $h_\epsilon$ и $h_\epsilon^\delta$:
    \[
        \|h_\epsilon - h_\epsilon^\delta\|_2 \leq = \sqrt {\int_{-\pi}^{\pi} |h_\epsilon - h_\epsilon^\delta|^2 dx} \leq \sqrt{2 \delta \cdot (2M)^2} = 2M\sqrt {2\delta}.
    \]
    Нам необходимо показать существование такого $\delta$, что $2M \sqrt {2\delta} \leq \frac{\epsilon}{3}$.
    Нам подойдёт всякая $\delta$ такая, что
    \[
        \delta < \frac{\epsilon^2}{2(6M)^2}.
    \]
    В то же время, по теореме Фейера, существует тригонометрический полином $T$ такой, что \[\|h_\epsilon^\delta - T\|_C \leq \frac{\epsilon}{3\sqrt {2\pi}}.\]
    Оценим $2$-норму их разности:
    \[
        \|h_\epsilon^\delta - T\|_2 \leq \sqrt {\int_{-\pi}^{\pi} (\sup|h_\epsilon^\delta - T|)^2dx} = \sqrt {2\pi}\|h_\epsilon^\delta - T\|_C \leq \dfrac{\epsilon}{3}.
    \]
    Применение неравенства треугольника даёт нам утверждение теоремы:
    \[
        \|f - T\|_2 \leq \|f - h_\epsilon\|_2 + \|h_\epsilon - h_\epsilon^\delta\|_2 + \|h_\epsilon^\delta - T\|_2 < \epsilon.
    \]
\end{proof}
