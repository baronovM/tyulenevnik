\begin{corollary}
    Все признаки которые были для рядов Фурье, для них есть аналоги:
    \begin{enumerate}
        \item Пусть $f \in \mathbb{L}^{loc}_1(\R)$ в некоторой $U_\delta(\underline{x})$ и имеет ограниченную вариацию, тогда:
        \[
            f(\underline{x}) = F^{-1} [F[f]] (\underline{x})
        \]
        \item Пусть $f \in \mathbb{L}_1(\R)$ удовлетворяет условию Гёльдера с степенью $\alpha \in (0; 1]$ в некоторой $U_\delta(\underline{x})$. Тогда:
        \[
            f(\underline{x}) = F^{-1} [F[f]] (\underline{x})
        \]
        \item (Условие Дини) Пусть $f \in  \mathbb{L}_1(\R)$ и $\exists \delta > 0$ и $\exists c > 0$ такие что:
        \[
            \int_0^\delta | \frac{f(\underline{x} - u)+f(\underline{x} - u)}{2} - C |  \frac{du}{u} < +\infty \]

        Тогда $F^{-1} [F[f]] (\underline{x}) = C$

    \end{enumerate}
\end{corollary}
\subsection{Преобразование Фурье свёртки}
\begin{reminder}
    Пусть $f, g \in L_1(\R^n)$. Свёрткой называют $f * g = \int\limits_{\R^n} f(x-t)g(t)dt$. $f*g$ корректно определенная измеримая функция. Она является интегрируемой, если $f$ и $g$ интегрируемы. Свертка обладает свойствами:
    \begin{itemize}
        \item Коммутативность. $f*g = g*f$
        \item Ассоциативность. $(f*g)*h = f(g*h)$
    \end{itemize}
\end{reminder}

\begin{theorem}
    Пусть $f,g\in L^1(\mathbb R)$. Тогда
    \[
        F[f * g](x) \;=\;\sqrt{2\pi}\, F[f](x)\,   F[g](x),
    \]
    где
    \[
        F[h](x)=\frac1{\sqrt{2\pi}}\int_{\mathbb R}e^{-i x y}h(y)\,dy.
    \]
\end{theorem}

\begin{proof}
    По определению
    \[
        F[f*g](x)
        =\frac1{\sqrt{2\pi}}\int_{\mathbb R}e^{-i x y}\bigl(f*g\bigr)(y)\,dy
        =\frac1{\sqrt{2\pi}}\int_{\mathbb R}e^{-i x y}
        \Bigl(\int_{\mathbb R}f(y-t)\,g(t)\,dt\Bigr)dy.
    \]
    Меняем порядок интегрирования и расписываем
    $e^{-i x y}=e^{-i x (y-t+t)}=e^{-i x t}e^{-i x (y-t)}$:
    \begin{align*}
        F[f*g](x)
        &=\frac1{\sqrt{2\pi}}
        \int_{\mathbb R}\int_{\mathbb R}e^{-i x t}e^{-i x (y-t)}f(y-t)\,g(t)\,dy\,dt\\
        &=\frac1{\sqrt{2\pi}}
        \int_{\mathbb R}g(t)e^{-i x t}
        \Bigl(\int_{\mathbb R}e^{-i x (y-t)}f(y-t)\,dy\Bigr)dt.
    \end{align*}
    Внутренний интеграл по $y$ при замене переменной $u=y-t$ даёт
    \[
        \int_{\mathbb R}e^{-i x (y-t)}f(y-t)\,dy
        =\int_{\mathbb R}e^{-i x u}f(u)\,du
        =\sqrt{2\pi}\, F[f](x).
    \]
    Следовательно
    \[
        F[f*g](x)
        =\frac1{\sqrt{2\pi}}\Bigl(\sqrt{2\pi}\, F[f](x)\Bigr)
        \int_{\mathbb R}g(t)e^{-i x t}\,dt
        =\sqrt{2\pi}\, F[f](x)\, F[g](x).
    \]
\end{proof}
\begin{note}
    Интеграл двойного интегрирования можно менять местами по теореме Фубини (а для неотрицательных функций по теореме Тонелли), поскольку
    \[
        \iint_{\mathbb R^2}\bigl|f(y-t)\bigr|\;\bigl|g(t)\bigr|\,dy\,dt
        =\int_{\mathbb R}\bigl|g(t)\bigr|\Bigl(\int_{\mathbb R}\bigl|f(y-t)\bigr|\,dy\Bigr)dt
        =\|g\|_{L^1}\,\|f\|_{L^1}<\infty.
    \]
    Таким образом все переходы с перестановкой интегралов обоснованы.
\end{note}



\begin{theorem}[преобразование Фурье производной]
    Пусть
    \[
        f\in L^1(\mathbb R)\cap C^1(\mathbb R),
        \quad f'\in L^1(\mathbb R),
    \]
    и существуют конечные пределы
    \[
        \lim_{x\to\pm\infty}f(x)=0.
    \]
    Тогда для любого $y\in\mathbb R$
    \[
        \mathcal F[f'](y)
        = i y\,\mathcal F[f](y),
    \]

\end{theorem}

\begin{proof}
    \textbf{Шаг 1.}
    Сначала докажем, что $\lim\limits_{x\to\pm\infty}f(x)=0$. По формуле Ньютона–Лейбница
    \[
        f(x)=f(0)+\int_{0}^{x}f'(t)\,dt.
    \]
    Так как $f'\in L^1(\mathbb R)$, то
    \[
        \lim_{x\to+\infty}f(x)
        =f(0)+\int_0^{+\infty}f'(t)\,dt =: A
    \]
    существует. Предположим $A\neq0$. Тогда найдётся $X_A$ такое, что при $x>X_A$ есть
    \[
        |f(x)|>\frac{|A|}{2}
        \quad\Longrightarrow\quad
        \int_{X_A}^{+\infty}|f(x)|\,dx = +\infty,
    \]
    что противоречит условию $f\in L^1(\mathbb R)$. Аналогичным рассуждением получаем и $\lim_{x\to-\infty}f(x)=0$.

    \medskip
    \noindent\textbf{Шаг 2.}
    Интегрируем преобразование Фурье $f'$ по частям:
    \begin{align*}
        \mathcal F[f'](y)
        &=\frac1{\sqrt{2\pi}}\int_{-\infty}^{+\infty}f'(x)e^{-i x y}\,dx\\
        &=\frac1{\sqrt{2\pi}}\Bigl[f(x)e^{-i x y}\Bigr]_{-\infty}^{+\infty}
        -\frac1{\sqrt{2\pi}}\int_{-\infty}^{+\infty}f(x)\bigl(-i y\bigr)e^{-i x y}\,dx\\
        &= i y\;\frac1{\sqrt{2\pi}}\int_{-\infty}^{+\infty}f(x)e^{-i x y}\,dx
        = i y\,\mathcal F[f](y).
    \end{align*}
    Поскольку $f(x)e^{-i x y}\Big|_{x\to\pm\infty}=0$, граничные слагаемые равны нулю.
\end{proof}
\begin{corollary}
    Пусть $k\in\mathbb{N}$,
    \[
        f,\;f',\;\dots,\;f^{(k-1)}\in L^1(\mathbb R)\cap C(\mathbb R),
        \quad
        f^{(k)}\in L^1(\mathbb R)\text{ и кусочно непрерывна на }\mathbb R.
    \]
    Тогда для каждого $y\in\mathbb R$
    \[
        \mathcal F\bigl[f^{(k)}\bigr](y)
        =\,(i y)^k\,\mathcal F[f](y),
    \]
    и при $|y|\to\infty$ выполняется
    \[
        \mathcal F[f](y)=o\bigl(|y|^{-k}\bigr).
    \]
\end{corollary}

\begin{proof}
    Докажем по индукции по $k$.

    \textit{База} ($k=1$) — это теорема о преобразовании Фурье производной:
    \[
        \mathcal F[f'](y)
        =iy\,\mathcal F[f](y).
    \]

    \textit{Шаг индукции.} Предположим, что для $k-1$ уже доказано
    \[
        \mathcal F\bigl[f^{(k-1)}\bigr](y)
        =(iy)^{\,k-1}\,\mathcal F[f](y).
    \]
    Поскольку $f^{(k)}\in L^1(\mathbb R)\cap C(\mathbb R)$ и $\lim_{x\to\pm\infty}f^{(k-1)}(x)=0$
    (аналогично первому доказательству), применяем к $f^{(k-1)}$ тот же приём интегрирования по частям:
    \[
        \mathcal F\bigl[f^{(k)}\bigr](y)
        =iy\,\mathcal F\bigl[f^{(k-1)}\bigr](y)
        =iy\,(iy)^{\,k-1}\,\mathcal F[f](y)
        =(iy)^k\,\mathcal F[f](y).
    \]
    Асимптотика. Из формулы
    \[
        \mathcal F[f^{(k)}](y)
        =(iy)^k\,\mathcal F[f](y)
    \]
    получаем
    \[
        \mathcal F[f](y)
        =(iy)^{-k}\,\mathcal F[f^{(k)}](y).
    \]
    Но $f^{(k)}\in L^1(\mathbb R)$, и по лемме Римана–Лебега
    $\mathcal F[f^{(k)}](y)\to0$ при $|y|\to\infty$. Значит
    \[
        \mathcal F[f](y)
        =o\bigl(|y|^{-k}\bigr),
        \quad |y|\to\infty.
    \]
\end{proof}


\begin{theorem}[Дифференцирование преобразования Фурье]
    Пусть
    \[
        f\in L^1(\mathbb R),
        \qquad
        x\,f(x)\in L^1(\mathbb R).
    \]
    Тогда функция
    \[
        \mathcal F[f](y)
        =\frac1{\sqrt{2\pi}}\int_{-\infty}^{+\infty}e^{-i x y}\,f(x)\,dx
    \]
    непрерывна и имеет непрерывную первую производную по параметру~$y$:
    \[
        \frac{d}{dy}\,\mathcal F[f](y)
        =\mathcal F\bigl[-\,i x\,f(x)\bigr](y).
    \]
\end{theorem}

\begin{proof}
    Зафиксируем конечный отрезок $[y_1,y_2]\subset\mathbb R$ и положим
    \[
        g(x,y)
        =\frac1{\sqrt{2\pi}}\,e^{-i x y}\,f(x).
    \]
    Тогда
    \[
        \mathcal F[f](y)=\int_{-\infty}^{+\infty}g(x,y)\,dx.
    \]
    Вычислим частную производную по $y$:
    \[
        \frac{\partial}{\partial y}g(x,y)
        =\frac1{\sqrt{2\pi}}\bigl(-i x\bigr)e^{-i x y}\,f(x).
    \]
    По условию $x\,f(x)\in L^1(\mathbb R)$, поэтому
    \[
        \biggl|\frac{\partial}{\partial y}g(x,y)\biggr|
        =\frac{|x\,f(x)|}{\sqrt{2\pi}}
        \quad\text{— интегрируемая функция (независимо от $y$).}
    \]
    Следовательно, по теореме о дифференцировании параметрического
    интеграла под знаком интеграла можно переставить дифференцирование и
    интегрирование:
    \[
        \frac{d}{dy}\,\mathcal F[f](y)
        =\frac{d}{dy}\int_{-\infty}^{+\infty}g(x,y)\,dx
        =\int_{-\infty}^{+\infty}\frac{\partial}{\partial y}g(x,y)\,dx
        =\mathcal F\bigl[-\,i x\,f(x)\bigr](y).
    \]
    При этом та же теорема гарантирует непрерывность производной по $y$,
    а из непрерывности $g(x,y)$ по $y$ и теоремы о переходе к пределу под интегралом следует непрерывность  $\mathcal F[f](y)$.
\end{proof}

\newpage
\section{Обобщенные функции}
На самом деле <<обобщённые функции>>~---~это не очень удачный перевод: правильнее было бы называть их \emph{распределениями}. Рассмотрим неформальную идею:

Существует базовая лемма Дю Буа--Реймона из вариационного исчисления, которая по сути говорит следующее: знать функцию <<поточечно>>~---~то же самое, что знать её действие на распределённые в пространстве объекты.

Почему термин <<распределение>>? Потому что мы обычно интегрируем \(f\), рассматривая её как плотность какой-то физической величины (скажем, плотность заряда или массы), а \(\varphi\) моделирует измерительный прибор. И любой реальный прибор не умеет «снимать» значение в точке~---~он измеряет величину в некотором объёме: например, мы не определяем температуру в математической точке, а получаем среднее по окружающему пространству.

Таким образом, мы пытаемся описать физический процесс поточечно, но лемма Дю Буа--Реймона показывает: это вовсе не обязательно~---~можно <<тестировать>> функцию любыми приборами произвольного размера, то есть рассматривать её действие на любые <<распределённые>> тест-функции.



\begin{lemma}[Дю Буа—Реймона]
    Пусть $f\in L^1_{\mathrm{loc}}(\mathbb R^n)$ и для любой $\varphi\in C_0(\mathbb R^n)$ - те непрерывна и имеет компактный носитель
    \[
        \int_{\mathbb R^n}f(x)\,\varphi(x)\,dx=0.
    \]
    Тогда $f=0$ почти всюду.
\end{lemma}

\begin{proof}
    \textbf{Шаг 1.}
    По теореме Лебега о дифференцировании интеграла почти всюду
    \[
        f(x)
        =\lim_{r\to0}\frac1{|B_r(x)|}\int_{B_r(x)}f(y)\,dy
        \tag{$*$}
    \]
    Пусть $E=\{x:f(x)\neq0\}$ имеет положительную меру.

    \textbf{Шаг 2.}
    Для $x\in E$ и $\varepsilon>0$ по $(*)$ существует $\delta>0$ такое, что
    \[
        \Bigl|\frac1{|B_r(x)|}\int_{B_r(x)}f(y)\,dy - f(x)\Bigr|<\varepsilon/2
        \quad
        \forall\,0<r<\delta.
    \]
    Возьмём тест-функцию
    \[
        \psi_r(y)
        =\frac1{|B_r(x)|}\,\psi\!\Bigl(\frac{y-x}{r}\Bigr),
    \]
    где $\psi\in C_0^\infty$, $\psi\ge0$, $\int\psi=1$, $\mathrm{supp}\,\psi\subset B_1(0)$.
    Тогда $\psi_r\in C_0^\infty$, $\int\psi_r=1$, $\mathrm{supp}\,\psi_r\subset B_r(x)$, и
    \begin{align*}
        \Bigl|\frac1{|B_r(x)|}\int_{B_r(x)}f(y)\,dy - \int f(y)\,\psi_r(y)\,dy\Bigr|
        &\le
        \frac1{|B_r(x)|}\int_{B_{r}(x)\setminus B_r(x)}|f(y)|\,dy
        <\varepsilon/2.
    \end{align*}

    \textbf{Шаг 3.}
    По условию $\int f\,\psi_r=0$, поэтому
    \[
        |f(x)| \le \Bigl|\frac1{|B_r|}\!\int_{B_r}f - \int f\,\psi_r\Bigr| + \varepsilon/2
        <\varepsilon.
    \]
    Так как $\varepsilon>0$ произвольна, получаем $f(x)=0$ для почти всех $x$.
\end{proof}

\subsection{Пространства $\D$ и $\D'$}

\begin{definition}[Пространство пробных функций]
    Пространством $\mathcal{D}(\mathbb{R}^n)$ назовем $C_0^\infty(\mathbb{R}^n)$ со введенной сходимостью:

    Последовательность $\{ \varphi_m\} \subset \mathcal{D}(\mathbb{R}^n)$ сходится к $\varphi \in \mathcal{D}(\mathbb{R}^n)$, то есть $\varphi_m \xrightarrow[\mathcal{D}(\mathbb{R}^n)]{} \varphi$, если
    \begin{enumerate}
        \item $\forall \alpha \in \mathbb{N}^n_0 \hookrightarrow D^{\alpha} \varphi_m \mathrel{\mathop{\rightrightarrows}\limits_{\mathbb{R}^n}} D^{\alpha} \varphi$, $m \to \infty$;
        \item $\exists C > 0$: $\text{supp}\varphi_m \subset B_c(0)$ $\forall m \in \mathbb{N}$.
    \end{enumerate}

    Здесь
    \[
        \alpha = (\alpha_1, \ldots, \alpha_n), \qquad
        |\alpha|=\alpha_1+\cdots+\alpha_n,
        \qquad
        D^\alpha
        =\frac{\partial^{|\alpha|}}{\partial x_1^{\alpha_1}\,\partial x_2^{\alpha_2}\,\cdots\,\partial x_n^{\alpha_n}}.
    \]
\end{definition}

\begin{definition}
    Назовем $\mathcal{D}'(\mathbb{R}^n)$~---~пространство всех линейных непрерывных функционалов на $\mathcal{D}(\mathbb{R}^n)$.
    Функционал
    \[
        T\in \mathcal{D}'(\mathbb{R}^n)
        \quad\Longleftrightarrow\quad
        T\text{: }\mathcal{D}(\mathbb{R}^n)\to\mathbb{R}
    \]
    и удовлетворяет двум условиям:
    \begin{enumerate}
        \item (\emph{Линейность})
        $\displaystyle
        T\bigl(\alpha\,\varphi_1 + \beta\,\varphi_2\bigr)
        =\alpha\,T(\varphi_1)+\beta\,T(\varphi_2),
        \quad
        \forall\,\alpha,\beta\in\mathbb{R},\;\forall\,\varphi_1,\varphi_2\in\mathcal{D}(\mathbb{R}^n).
        $
        \item (\emph{Непрерывность})
        $\displaystyle
        \forall \{\varphi_m\} \subset \mathcal{D}(\mathbb{R}^n)\text{: }\exists \varphi \in \mathcal{D}(\mathbb{R}^n)\text{: } \varphi_m \xrightarrow[\mathcal{D}(\mathbb{R}^n)]{} \varphi
        \;\hookrightarrow \;
        T(\varphi_m)\;\longrightarrow\;T(\varphi), \ m\to\infty.
        $
    \end{enumerate}
\end{definition}

\begin{remark}
    Далее <<действие>> функционала будем обозначать не как $T(\varphi)$, а как $\langle T, \varphi \rangle$.
\end{remark}

\begin{definition}
    Пусть $\varphi\in\mathcal{D}'(\mathbb{R}^n).$ Говорят, что \(\varphi\) является \emph{регулярным} обобщённым функционалом (или регулярной обобщённой функцией), если
    \[
        \exists\,f_\varphi\in L^{loc}_{\mathrm{1}}(\mathbb{R}^n)\quad\text{такое, что}\quad
        \langle \varphi,\psi\rangle
        =\int_{\mathbb{R}^n}f_\varphi(x)\,\psi(x)\,dx,
        \quad
        \forall\,\psi\in\mathcal{D}(\mathbb{R}^n).
    \]
    В противном случае \(\varphi\) называют \emph{сингулярным} распределением.
\end{definition}
