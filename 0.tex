\section{Функции нескольких переменных}

\setlength{\epigraphwidth}{0.5\textwidth}
\epigraph{\normalsize{Мы будем работать по бразильской системе:\\Я расскажу столько, сколько захочу, \\ Вы поймете столько, сколько сможете}}{\large{© А. И. Тюленев}}


\subsection{Предел функции нескольких переменных}

Будем использовать <<тензорные>>$\ $обозначения, к примеру, $x^0 _1$, $0$~---~это индекс, а $1$ означает номер координаты.

Ранее было введено понятие предела и непрерывности по множеству для для отображения $f$: $X \mapsto Y$, где $X = (X, \rho)$ и $Y = (Y, d)$ метрические пространства.

\begin{definition}
    \textit{(Последовательность Гейне в точке)} Последовательностью Гейне в точке $x^0$ называют последовательность $\{x^n\}$:

        \begin{enumerate}
        \item $\lim\limits_{n\to \infty} x^{n} = x^{0};$
        \item $x^{n} \neq x^{0} \  \forall n \in \N.$
    \end{enumerate}
\end{definition}

\begin{definition}
    Пусть $x^{0}$~---~предельная точка множества $E \subset X$. Пусть $Y = (Y, d)$ - метрическое пространство и $f: E \mapsto Y$ 
    Будем говорить, что $\lim\limits_{\underset{x \in E}{x\to x_{0}}} f (x) = y^{0} \in Y$, если:

    $$ \forall \epsilon > 0 \  \exists \delta =  \delta (\epsilon) > 0: \forall x \in \mathring{B}_{\delta (\epsilon)} (x^{0}) \cap E \hookrightarrow f (x) \in B_{\epsilon} (y^{0}) \cap E \text{~---~по Коши.}$$

    

    $$ \forall \text{ последовательности Гейне }\{ x^{n}\} \subset E \  \text{в точке } x_{0} \hookrightarrow  \exists  \lim\limits_{n\to \infty} f (x^{n}) = A\text{~---~по Гейне.}$$
        
\end{definition}

Далее рассмотрим функции на $X = \R ^n$:
% $ \rho = \sqrt{\sum_{k = 1}^{n} (x_i - y_i)^2}$
% $\R^n = {x = (x_1,...,x_n): x_i \in \R}$

\begin{definition}
    \textit{(Предел по совокупности)} Пусть $f$: $\mathring{B}_{\delta _0} \mapsto \R$. Предел в смысле общей теории метрических пространств будем называть пределом по совокупности переменных и записывать:

    $$\lim_{x \to x^0} f(x) = A \in \overline{\R}  \Longleftrightarrow \lim\limits_{\underset{x_n \to x_n^0}{x_1 \to x_{1}^0}} f (x_1, ..., x_n) = A. $$
    
    $$\forall \epsilon > 0 \  \exists \  \delta(\epsilon) > 0: \ \forall x = (x_{1}, ..., x_n) \in \mathring{B}_{\delta(\epsilon)}(x^0) \  \hookrightarrow \ f(x) \in U_\epsilon(A).
    $$
    
\end{definition}

\begin{note}
    Даже в случае абстрактных метрических пространств отображение $f$: $X \mapsto Y$, где $X = (X, \rho)$ и $Y = (Y, d)$ --- абстрактные метрические пространства. Определния предела по Коши и по Гейне эквивалентны
\end{note}

\begin{definition}
    \textit{(Предел по направлению)} Пусть $f$: $B_{\delta _0} \mapsto \R$ и $\overline{l} \in \R^n$: $||\overline{l}|| = 1$. Пределом функции $f$ в точке $x^0$ по направлению $\overline{l}$ будем называть:

    $$\lim_{t \to +0} f (x^0 + t\overline{l}) = y^{0} \in Y. $$
\end{definition}

\begin{definition}
$S_r^{n-1}(x^0) := (n-1)$мерная сфера радиуса $r$ с центром в точке $x^0$.

    Сфера в $\R^n$ имеет размерность $n-1$, так как нужна $n-1$ координата для определения положения точки, ведь радиус фиксирован.


\end{definition}

\begin{lemma}
    Если существует предел по совокупности переменных, то существуют пределы по любому направлению и все они равны пределу по совокупности

    Пусть $f$: $B_{\delta _0}(x^0) \mapsto \R$ и $\exists \lim_{x \to x^0} f(x) = A \in \overline{\R}$ Тогда $\forall \overline{l} \in S^{n-1}_1(0)$: $ \exists \lim_{t \to +0} f(x^0 + t\overline{l}) = A.$ 
    
\end{lemma}

\begin{proof}
    $$\exists \lim_{x \to x^0} f(x) = A \in \overline{\R} \Longrightarrow    \forall \epsilon > 0 \  \exists \  \delta(\epsilon) > 0: \ \forall x \in \mathring{B}_{\delta(\epsilon)}(x^0) \  \hookrightarrow \ f(x) \in U_\epsilon(A).
    $$

    $$\forall \epsilon > 0 \  \exists \  \delta(\epsilon) > 0: \forall t \in (0, \delta(\epsilon)) \hookrightarrow x^0 + t \overline{l} \in \mathring{B}_{\delta}(x^0) \Longrightarrow f(x^0 + t \overline{l}) \in U_{\epsilon}(A).$$

    Примечание: $\delta(\epsilon)$ в определении второго предела такое же, как в первом.
\end{proof}


\begin{note}
    Из существования предела по любому направлению не следует существвование предела по совокупности.

    Более того пределы по всем направлениям могут быть равны, а предела по совокупности может всё равно не существовать

\end{note}

\begin{example}$\ $
    \sidefig(8 cm)(7 cm)
{
    \begin{flushleft}
        \normalsize
        \centering
        Положим $ z(x, y) =
        \begin{cases}
            1, \ y = x^2, x \neq 0\\
            0, \text{иначе}&
        \end{cases}  
        $
    \end{flushleft}
}
{
\begin{tikzpicture}
%    \begin{axis}[
%        view={30}{30}, % Adjusts the angle at which you %view the 3D plot
%        xlabel=$x$, ylabel=$y$, zlabel=$z$,
%        domain=-10:10, % Domain for your parameter t
%        ]
    \begin{axis}[
        view={25}{30},
        xmin=-5, xmax=5,
        ymin=-0.1, ymax=7, 
        zmin=-0.1, zmax=2,
        xlabel=$x$, ylabel=$y$, zlabel=$z$,
        axis lines=middle,
        axis on top,
    ]
    \tikzset{arrowstyle/.style={decoration={
        markings,
        mark=at position #1 with {\arrow{>}}},postaction={decorate}}}
    \addplot3[->, blue] coordinates {(0,0,0) (1,0,0)};
    \addplot3[->, red] coordinates {(0,0,0) (0,1,0)};
    \addplot3[->, green] coordinates {(0,0,0) (0,0,1)};
    % Plot your parametric equation
    \addplot3[samples=400, domain=-5:-0.45, samples y=0, -, thick, arrowstyle=1] ({-0.5*x}, {0.25*x^2}, {1});
    \addplot3[samples=400, domain=5:0.45, samples y=0, -, thick, arrowstyle=1] ({-0.5*x}, {0.25*x^2}, {1});
    % Draw a point at (0, 0, 0)
    \node at (axis cs:0,0,0) [circle, fill, inner sep=1.5pt]{};  
%    \node at (axis cs:0,0,1) [circle, fill=white, inner sep=1.8pt]{};
%    \draw[thick, <-] (0.03, 0.03, 0.992) arc(-70:-60:2);
%    \draw[thick, <-] (0, 0, 0.995) arc(80:30:-1.15);

\end{axis}
\end{tikzpicture}

}

    Чтобы доказать отсутствие предела, воспользуемся определением по Гейне. Достаточно привести две последовательности Гейне, на которых получаются разные пределы.

    \begin{enumerate}
        \item $ \left\{ \left(\frac{1}{n}, \frac{1}{n^2}\right) \right\} - \text{последовательность Гейне в }(0, 0); \  z\left(\frac{1}{n}, \frac{1}{n^2}\right) = 1 \to 1, n \to \infty.
        $
        \item $ \left\{ \left(0, \frac{1}{n}\right) \right\} - \text{последовательность Гейне в }(0, 0); \  z\left(\frac{1}{n}, \frac{1}{n^2}\right) = 0, \  \forall n \in \N.
        $
    \end{enumerate}

    Мы предъявили две последовательности Гейне в точке $(0, 0)$, пределы которых не равны, значит, предела по совокупности нет.
    
    Докажем, что существует предел по любому направлению. \


\sidefig(8 cm)(7 cm)
{
    \begin{flushleft}
    \normalsize
    Перейдем в полярные координаты. Тогда любая точка имеет вид $(x, y) = (r \cos\phi, r \sin\phi)$. Выберем направляющий единичный вектор $\overline{l} = (\cos\phi, \sin\phi), \ \phi \in [ - \pi, \pi ).$
    \end{flushleft}
}
{
    \begin{tikzpicture}[scale=0.9]
    \begin{axis}[
        axis lines=middle,
        xlabel=$x$, ylabel=$y$,
        xmin=-3.1, xmax=3.1,
        ymin=-0.175, ymax=9,
        xtick={-2,-1,...,2}, % Custom x-axis ticks
        ytick={2,4,...,8},   % Custom y-axis ticks
        tick label style={font=\small},
    ]
    % Decoration style for arrows
    \tikzset{arrowstyle/.style={decoration={
        markings,
        mark=at position #1 with {\arrow{>}}},postaction={decorate}}}
    
    % Draw the parabola with arrows lying on it
    \addplot[samples=100, domain=-3:-0, ultra thick, arrowstyle=0.98, name path=parabola] {x^2};
    \addplot[samples=100, domain=3:0, ultra thick, arrowstyle=0.98] {x^2};
    \node at (axis cs:0,0) [circle, fill=white, inner sep=3pt]{};
    % Draw a line crossing the parabola, named path for intersection calculation
    \addplot[domain=0:3, ultra thick, red, name path=crossline] {2.5*x};
    \addplot[->, ultra thick, red] coordinates {(0,0) (1.6,4)};
    \addplot[domain=-0.5:0.5, thin, black] {0};
    \addplot[domain=0:0.001, thin, black] {10000*x};
    
    \end{axis}


\end{tikzpicture}

}

    $\text{Если }\phi \in [ - \pi, 0 ]\text{, то }\forall r \geq 0, \ z(r \cos\phi, r \sin\phi) = 0 \Longrightarrow \exists \lim_{r \to +0} z(r \cos\phi, r \sin\phi) = 0$

    $\text{Если } \phi \in ( 0, \pi ) \text{, то есть точка пересечения с параболой: } r_0 = \frac{\sin\phi}{\cos^2\phi}\Longrightarrow  $ 

    \begin{flushright}
$\Longrightarrow  \forall\rho \in (0, r_0) \hookrightarrow \ \exists \lim_{\rho \to +0} z(r \cos\phi, r \sin\phi) = 0.$        
    \end{flushright}


    То есть по любому направлению есть предел и он равен 0, но предела по совокупности нет.

\end{example}

\subsection{План исследования функции двух переменных на существование предела в точке}

Пусть $f:$ $B_{r}((x^0, y^0)) \mapsto \R.$

\begin{enumerate}
    \item Ищем пределы по направлению, то есть $\forall \phi \in [0, 2\pi)$ ищем предел
$$\lim_{r \to +0} f(x^0 + r \cos\phi, \ y^0 +  r \sin\phi) \ (*)$$

Если при каком-то $\phi$ предела (*) не существует, либо при разных $\phi$ пределы разные, то предела $f(x, y)$ в точке $(x^0, y^0)$ по совокупности переменных не существует. Так как это необходимое условие (но не достаточное).



Пусть оказалось, что:

$$\forall \phi \in [0, 2\pi): \  \exists \lim_{r \to +0} f(x^0 + r \cos\phi, \ y^0 +  r \sin\phi) = A^{*} \in \R.$$

\begin{proposition}(Метод полярных координат)

    Пусть $\exists \delta_0 > 0 \text{ и функция } \rho(r)$: $ [0, \delta_0) \mapsto [0, +\infty),\text{ называемая \textit{мажорантой}:}$
    $$
    \begin{cases}
        \forall r \in [0, \delta_0) \text{ и } \forall \phi  \in [0, 2\pi) \hookrightarrow |f(x^0 + r \cos\phi, \ y^0 +  r \sin\phi) - A^{*}| \leq \rho(r), \\
        \lim_{r \to +0} \rho(r) = 0.&
    \end{cases}  
    $$

    Это равносильно     $\exists \lim_{(x, y) \to (x^0, y^0)} f(x^0 + r \cos\phi, \ y^0 +  r \sin\phi) = A^{*}  $.

\begin{proof}
    $(\Longrightarrow)$Распишем определение предела функции:
    $$\forall \epsilon > 0 \  \exists \  \delta_1(\epsilon) > 0: \ \forall (x, y)  \in \mathring{B}_{\delta_1(\epsilon)}(x^0, y^0) \  \hookrightarrow \ |f(x) - A^{*}| < \epsilon.
    $$

Распишем определение предела мажоранты:

$$\forall \epsilon > 0  \ \exists \delta_2(\epsilon) \in (0, \delta_0):  \forall r \in (0, \delta_2(\epsilon)) \hookrightarrow 0 \leq \rho(r) < \epsilon.$$

Заметим, что точка $(x, y)$ в полярных координатах может быть записана:

$    \begin{cases}
        x = x^0 + r \cos\phi\\
        y = y^0 +  r \sin\phi&
    \end{cases}$

    $(x, y) \text{ пробегает шар } \mathring{B}_{\delta(\epsilon)}(x^0, y^0) \Longleftrightarrow 
        \begin{cases}
        r \in (0, \delta(\epsilon))\\
        \phi \in [0, 2\pi)&
    \end{cases}
    $

    В итоге:
    $$\forall \epsilon > 0  \ \exists \delta_2(\epsilon) \in (0, \delta_0): \forall r \in \delta(\epsilon) \text{ и } \phi \in [0, 2\pi) \hookrightarrow |f(x^0 + r \cos\phi, \ y^0 +  r \sin\phi) - A^{*}| \leq \rho(r) < \epsilon,$$

    Что равносильно $\ \forall (x, y) \in \mathring{B}_{\delta(\epsilon)}(x^0, y^0) \hookrightarrow |f(x, y) - A^{*}| < \epsilon.$

    $(\Longleftarrow)$ Теперь докажем, что если есть передел по совокупности, то существует мажоранта.

    $\exists \lim_{(x, y) \to (x^0, y^0)} f(x^0 + r \cos\phi, \ y^0 +  r \sin\phi) = A^{*}  \Longleftrightarrow$
    \begin{flushright}
            $\Longleftrightarrow \forall \epsilon > 0 \  \exists \delta(\epsilon) > 0: \ \forall (x, y)  \in \mathring{B}_{\delta(\epsilon)}(x^0, y^0) \  \hookrightarrow \ |f(x, y) - A^{*}| < \epsilon \Longrightarrow
    $
    \end{flushright}

    Тогда $\forall \epsilon > 0  \ \exists \delta(\epsilon) \in (0, \delta_0):  \forall r \in (0, \delta(\epsilon)) \ \forall \phi \in [0, 2\pi) \hookrightarrow$
    \begin{flushright}
        $\hookrightarrow |f(x^0 + r \cos\phi, \ y^0 +  r \sin\phi) - A^{*}|< \epsilon$
    \end{flushright}

    $\forall \epsilon > 0\ \exists \delta(\epsilon) \in (0, \delta_0):  \forall r \in (0, \delta(\epsilon)) \  \hookrightarrow $
    \begin{flushright}
            $ \hookrightarrow 
    \hat{\rho}(r) :=
    \underset{\phi \in [0, 2\pi)}{\sup}|f(x^0 + r \cos\phi, \ y^0 +  r \sin\phi) - A^{*}| \leq \epsilon$
    \end{flushright}

    Тем самым определена функция $\hat{\rho}(r)$: $(0, \delta_{0}) \mapsto [0, +\infty)$, которую мы и назовем мажорантой. Ведь $ \forall \phi \in [0, 2\pi) \hookrightarrow f(x, y) \leq \hat{\rho}(r)$ и $\lim_{r \to +0} (\hat{\rho(r)}) = 0.$
    
\end{proof}

% 1:24:00 Я ЗАЕБАЛСЯ СУКА ОН НОЛЬ СЛОВ ГОВОРИТ ЭТО КАКАЯ_ТО ХУЙНЯ
\end{proposition}
\end{enumerate}


\subsection{Повторные пределы}

По определеню двумерный куб $Q_{\delta}(x^0, y^0) := (x^0 - \delta, x^0 + \delta) \times (y^0 - \delta, y^0 + \delta)$.

\begin{definition} (Повторный предел)
    
Пусть $f$: $Q_{\delta}(x^0, y^0) \mapsto \R$. Предположим, что $\forall y \in (y^0 - \delta, y^0 + \delta)$

$\exists \lim_{x \to x^{0}} f(x, y) = \phi(y).$

Тогда \textit{повторным пределом} функции $f$ в точке называют: $\lim_{y \to y^0} \phi(y) = \lim_{x \to x^0} \lim_{y \to y^0} f(x, y).$
\end{definition}

\begin{note}
    Повторный предел зависит от порядка переменных, по которым берется предел, то есть $ \exists \lim_{x \to x^0} \lim_{y \to y^0} f(x, y) = A_1 \in \R \neq \exists \lim_{y \to y^0} \lim_{x \to x^0} f(x, y) = A_2 \in \R$
\end{note}

\begin{note}

    Из существования предела по совокупности не следует существование повторного предела, и наоборот, даже если повторные переделы равны.
\end{note}

\begin{example}
    $ f(x, y) =
    \begin{cases}
        (x + y)\sin\frac{1}{x} \sin \frac{1}{y}, \ xy \neq 0\\
        0, \text{иначе}&
    \end{cases}  
    $

    Метод полярных координат: $|f(r \cos\phi, \ r \sin\phi)| \leq |x| + |y| \leq r(|\cos\phi| + |\sin \phi|) \leq|r|$, значит предел по совокупности равен нулю. Повторного же предела ни по одной из переменных не существует.

    %ВСТАВИТЬ ТЕОРЕМУ ОТ СКУФА
\end{example}


\begin{note}
    Повторные пределы могут существовать и быть равны, а предела по совокупности модет не быть, пример (Парабола, поднятая вверх) $ z(x, y) =
    \begin{cases}
        1, \ y = x^2\\
        0, \text{иначе}&
    \end{cases}  
    $
\end{note}




\begin{theorem} (Непрерыность композиции отображений)
Пусть даны три метрических пространства $(X, d_{1}), \ (Y, d_2), \ (Z, d_3)$. В них выбораны подмножества $G \subset X, \ E \subset Y, F \subset Z$. И есть функции:

$$f: G \mapsto E$$
$$h: E \mapsto H$$


Пусть $\phi$ непрерывна в точке $x^0 \in G$ по множеству $G$. Пусть  $h$ непрерывна в точке $y^0 \in E$ по множеству $E$. При том $y^0 = \phi(x^0)$

Тогда композиция $f \circ \phi: G \mapsto F$ непрерывна в точке $x^0$
\end{theorem}

\begin{proof}
    Доказательство идено такое же, как для функции одной переменной.

    Определение непрерывности функции $h$ в точке $y^0$:
    
$$ \forall \epsilon > 0 \  \exists \delta (\epsilon) > 0: \forall y \in B_{\delta (\epsilon)} (y^{0}) \cap E \hookrightarrow d_{3}(h(y),\  h(y^0)) < \epsilon$$

Определение непрерывности функции $f$ в точке $x^0$:
    
$$ \forall \delta > 0 \  \exists \sigma (\delta) > 0: \forall y \in B_{\sigma (\delta)} (x^{0}) \cap G \hookrightarrow d_{2}(f(x),\  f(x^0)) < \delta$$

Тогда:

    
$$ \forall \epsilon > 0 \  \exists \widetilde{\sigma} =  \sigma(\delta(\epsilon)): \forall x \in B_{\widetilde{\sigma} (\delta)} (x^{0}) \cap G \hookrightarrow d_{2}(f(x),\  f(x^0)) < \delta(\epsilon) \Longrightarrow d_3(h(f(x)), \ h(f(x^0))) < 0$$

Следовательно композиция $h \circ f$ непрерывна в точке $X^0$

\end{proof}

\begin{definition}
    Функция $f: X \mapsto Y$, где $X$ и $Y$ - метрические пространства непрерына на $X$ если она непрерына по множеству в кажой точке $X$
\end{definition}

\begin{definition}
    Образом множества называют $f(X):= \{f(x): x \in X\}$
\end{definition}

\begin{theorem} (О непрерывном отображении компакта)

    Пусть $K \subset X$ - компакт. $(X, d)$ и $(Y, \rho)$ - метрические пространства. $f: K \mapsto Y$ - непрерывна на $K$. Тогда образ компакта $f(K)$ - компакт
\end{theorem}

\begin{proof}
    Рассмотрим произвольную последовательность точек в образе $\{y^n\} \subset f(K) $. Так как у каждой точки есть праобраз, мы имеем последовательность в компакте $\exists \{ x^n \} \subset K: \ f(x^n) = y^n, \  \forall n \in \N$. Так как она лежит в компакте, у нее есть сходящаяся  подпоследовательность $\{x^{n_{j}}\} \subset K$ к $\exists x^0 \in K$ такая, что $\lim_{j \to \infty} x^{n_{j}} = x^{0}$

    Но тогда в силу непрерывности $f$ на компакте, и как следствие в точке $x^0$. Тогда по Гейне существует предел: $\lim_{j \to \infty} f(x^{n_{j}}) = f(x^{0}) \in K \Longleftrightarrow \lim_{j \to \infty} y^{n_{j}} = y^0$

    То есть получили точку $y^0 \in f(K)$ и получили подпоследовательность $\{ y^{n_{j}} \}$ произваольной последовательности $\{ y^n \}$. Такую, что $\lim_{j \to \infty} y^{n_{j}} = y^0 \in f(K) $
    
\end{proof}


\begin{corollary}

    Если $(X, d), \ (Y, \rho)$~---~метрические пространства. $R \subset X$~---~компакт. А $f: \ K \mapsto Y$~---~непрерывное отображение. Тогда $f(K)$~---~ограниченное множество

    \begin{proof}
        Ранее было доказано, что компакт в $R^{n}$~---~замкнутое и вполне ограниченное множество. Также из предыдущей теоремы $f(K)$~---~компакт.
    \end{proof}
\end{corollary}

\begin{corollary} (Продвинутая теорема Вейерштрасса/О достижимости точных граней для функций непрерывных на компакте)

    Пусть $(X, \rho)$~---~метрические пространства. $K$~---~компакт и $f$: $K \mapsto \R$

    Если $f$ непрерывна на $K$, то она достигает на нем своего минимума и максимума, то есть 
    $$\exists \ min \{ f(x): x \in K \}$$
    $$\exists \ max \{ f(x): x \in K \}$$

\begin{proof}
    Так как $f(K) \subset \R$~---~компакт. Ранее было доказано что на $\R$ компакт замкнут и ограничен $\Longrightarrow$ Из определения инфимума $inf f(K) = m$ и $\forall \epsilon > 0: \ U_{\epsilon} \cap f(K) \neq \emptyset$ тогда по определению инфимум принадлежит замыканию $\inf f(k) \in cl f(K)$. Максимум аналогично
\end{proof}
\end{corollary}
